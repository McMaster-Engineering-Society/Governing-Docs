% % !TEX program = xelatex

% \documentclass{article}
% \usepackage[letterpaper, portrait, margin=1in]{geometry}
% \usepackage{hyperref}
% \usepackage{alphalph}

% \setcounter{secnumdepth}{5}
% \renewcommand{\labelenumi}{\alphalph{\value{enumi}})}
% \renewcommand{\labelenumii}{\alph{enumii}.}
% \renewcommand{\labelenumiii}{\roman{enumiii}.}

% \begin{document}

\hypertarget{financial-policies}{%
 \section{\texorpdfstring{\emph{FINANCIAL
     POLICIES}}{FINANCIAL POLICIES}}
 \label{financial-policies}}

\hypertarget{budget}{%
 \subsection{Budget}
 \label{budget}}
\begin{enumerate}
 \item
  The VPF should present the preliminary budget at the first MES Council
  meeting of the first term. This budget will contain the proposed
  accounts for the upcoming year and be updated to show any accounts
  used in between May and September.
 \item
  The VPF should make an effort to consult with the relevant
  stakeholders when preparing the budget.
 \item
  In the event of a non-mandatory MES membership fee, the VPF should
  make an effort to prepare for a variety of opt-out rates when
  preparing the budget.
 \item
  The VPF shall motion in the first meeting with the new council in
  April to approve the Welcome Week and Executive Operations budget so
  that operations can continue throughout the summer. This cumulative
  budget shall not exceed the funds reserved for Financial Contingency.
 \item
  The MES Council will approve the budget at the first MES Council
  meeting in September; this approval authorizes the MES Executive to
  make necessary expenditures outlined in the budget without further
  approval.
 \item
  Changes to the approved budget:

  \begin{enumerate}
   \item
    The MES Council can amend the budget by vote during the year.
   \item
    Expenditures not outlined in the budget must be approved according
    to the MES Funding Policies (see MES Bylaws Section H.3)
   \item
    Budgetary changes will be moved via motion accompanied by the
    following:

    \begin{enumerate}
     \item
      Reason for the change.
     \item
      Why and which budget should be changed to accompany the proposed
      expenditure.
    \end{enumerate}
  \end{enumerate}
 \item
  Reserve funds shall be set according to the MES Bylaws Section H.4.3.
 \item
  Contingency funds shall be set according to the MES Bylaws Section
  H.4.4.

\end{enumerate}

\hypertarget{payment-policies}{%
 \subsection{Payment Policies}
 \label{payment-policies}}
\begin{enumerate}
 \item
  Documentation for approved expenses, including an expense report
  (Appendix E), must be submitted to the VPF by March 30th of the given
  term. Expenses from the month of April will be reported similarly
  before April 30th.
 \item
  Expenses of Program Societies, MES Groups, or MES Teams, and
  individuals qualifying and approved for reimbursement will receive
  their funding via electronic funds transfer (EFT) to their respective
  bank account or via cheques written for approved expense reports.
 \item
  At the discretion of the VPF, payment for an invoice shall be done for
  an approved expense request.
 \item
  The VPF shall ensure that all expenses include sufficient and honest
  documentation, in accordance with the funding request as mentioned in
  the Policy Manual.
 \item
  Payment for sponsorship funding requests may be dispensed at the
  discretion of the MES Executive.
 \item
  All reimbursements must be approved by a minimum of two signing
  authorities on the MES bank account. These authorities include the
  VPF, VPSL, and President. Approval may take the form of a signature on
  a cheque, or digital approval of an EFT payment.
 \item
  Gasoline/mileage reimbursement for eligible travel (determined by the
  VPF) expenses will be \$0.22 per kilometer. Supporting map data
  stating total shortest distance travelled must be submitted.
 \item
  Any MES Member may access MES financial records during VPF office
  hours.
 \item
  The MES credit card shall be used to make advance payments for large
  expenses or to make large online payments to avoid having MES members
  or affiliated parties float large amounts of personal money. ``Large''
  is to be determined at the discretion of the VPF.

  \begin{enumerate}
   \item
    The VPF shall be the only one allowed to use the credit card to pay
    for online expenses. Regardless of how a purchase is made, the VPF
    shall be the only person to hold custody of the card and its payment
    information.
   \item
    The party wishing to have an expense charged to the MES Credit Card
    shall submit an explanation of what the item is for/how it will be
    used, and an Expense Report (Appendix E) for the purchase, along
    with an attached screenshot of the item and its price.
  \end{enumerate}
\end{enumerate}

\hypertarget{funding-policies}{%
 \subsection{Funding Policies}
 \label{funding-policies}}

\hypertarget{other-funding-requests}{%
 \subsubsection{Other Funding
  Requests}
 \label{other-funding-requests}}
All funding requests unlisted under the approved budget must be approved
by the appropriate level:

\begin{enumerate}
 \item
  Greater than \$1,500: General Meeting majority vote
 \item
  \$501 to \$1,500: MES Council majority vote
 \item
  \$500 or less: MES Executive majority vote

\end{enumerate}

\hypertarget{societies-funding-policy}{%
 \subsubsection{Societies Funding
  Policy}
 \label{societies-funding-policy}}
\begin{enumerate}
 \item
  A systematic method for the distribution of MES funds to all
  recognizedProgram Societies, in order to avoid overlap with MSU
  funding, to encourage continuity for each society from year to year
  and to develop accountability to the MES for the funds awarded.
 \item
  Each society is required to submit two packages to the AVPC of the MES
  each year: a Budget Proposal Package and a Final Report Package.
 \item
  Unless otherwise written by the VPF, the Budget Proposal Package will
  be submitted at least 10 business days before the first General
  Meeting in a given academic year and will contain:

  \begin{enumerate}
   \item
    A budget proposal (following the template provided on the MES
    website or available from the VPF)
   \item
    A list and description of all proposed events for the coming year
   \item
    A statement of the society's starting balance (carry-over from the
    previous year)
   \item
    Financial \& bank statements of the previous academic year
   \item
    Program Society Funding Form (Appendix M)
  \end{enumerate}
 \item
  Unless otherwise written by the VPF, The Final Report Package will be
  submitted at least 10 business days before the second General Meeting
  in a given academic year and it will contain:

  \begin{enumerate}
   \item
    An update of the original proposed budget
   \item
    A quantitative description of the distribution of MES funds
   \item
    One ``Event Summary Sheet'' for each event organized during the year
   \item
    A report of future recommendations and the overall functioning of
    the society
   \item
    Bank statement to support the updated budget
   \item
    Program Society Funding Form (Appendix M)
  \end{enumerate}
 \item
  The AVPC shall review the budget proposals presented and suggest an
  amount to be awarded to each society. The MES VPF will review the
  suggested funding for each society and decide on the final amount that
  each society is to receive.
 \item
  The budget proposals will be assessed according to the following
  criteria:

  \begin{enumerate}
   \item
    Starting balance
   \item
    Type and variety of activities offered
   \item
    Level of activity
   \item
    Level and Sources of Sponsorship
   \item
    Level of fundraising activity
   \item
    Amount of funding requested from the MES
   \item
    Approximate student population, with the input of a
    Programadministrator.
  \end{enumerate}
 \item
  Each Program society will receive a standard funding in the amount of
  \$2 per enrolled student. Each Program Society will receive a minimum
  of \$100 should they meet the quorum requirements at the first General
  Meeting of the MES. No Program Society will receive annual funding in
  excess of \$1000. Half of the annual funding allotment for each
  society is to be dispensed at the first General Meeting, with the
  remainder dispensed at the second General Meeting, except in the case
  of h) below.
 \item
  Program Societies not submitting a satisfactory budget to the AVPC by
  the deadline for the fall General Meeting shall be penalized one third
  of their society's annual allotment.
 \item
  Program Societies failing to meet quorum requirements at either
  General Meeting will not receive the funding allotted for that
  meeting. (Note: This is not classed as a penalty.)
 \item
  Societies failing to meet quorum requirements at the first General
  Meeting will still be eligible to receive funding (only up to the
  first half of their annual allotment) based upon individual motions
  presented to and voted upon by the MES Council.
 \item
  Funds accumulated from penalized societies will be reallocated into
  the general MES Societies budget whereupon it can be distributed at
  the second General Meeting. The distribution of the reallocated funds
  will be at the discretion of the MES Executive. Program
  Representatives will be notified in a timely fashion of the amount
  available.
 \item
  Funds will be allocated to each society via cheque or EFT from the
  VPF. The onus is on each Program Representative to pick up their
  cheques once notified by email. Failure to deposit these funds by the
  end of the current academic semester will result in the cheque being
  deemed null and void by the MES, and not replaceable under any
  circumstance.

\end{enumerate}

\hypertarget{clubs-and-teams-funding-policy}{%
 \subsubsection{Clubs and Teams Funding
  Policy}
 \label{clubs-and-teams-funding-policy}}
\begin{enumerate}
 \item
  A systematic method for the distribution of MES funds to all
  recognized MES Clubs and Teams, to avoid overlap with MSU funding,
  encourage continuity, and develop accountability to the MES, should be
  implemented.
 \item
  Each MES Club or Team is required to submit two packages to the AVPC
  each year: a Budget Proposal Package and a Final Report Package.
 \item
  Unless otherwise written by the VPF, the Budget Proposal Package will
  be submitted by July 31st and it will include:

  \begin{enumerate}
   \item
    A budget proposal (following the template provided on the MES
    website or available from the VPF)
   \item
    A list and description of all proposed events for the coming year
   \item
    A statement of the MES Club or Team's starting balance (carry-over
    from the previous year)
  \end{enumerate}
 \item
  Unless otherwise determined by the VPF, the Final Report Package will
  be submitted at least 10 business days before the second General
  Meeting in a given academic year and it will contain:

  \begin{enumerate}
   \item
    An update of the original proposed budget
   \item
    A quantitative description of the distribution of MES Funds
   \item
    One event summary sheet for each event organized during the year
   \item
    A report of future recommendations and the overall functioning of
    the MES Club or Team (the VPF will use these figures to propose the
    following year's budget)
  \end{enumerate}
 \item
  The budget proposals will be assessed according to the following
  criteria:

  \begin{enumerate}
   \item
    Starting balance
   \item
    Type and variety of activities offered to students
   \item
    Level of activity
   \item
    Performance at the competitions they take part in
   \item
    Level and Sources of Sponsorship
   \item
    Level of fundraising activity
   \item
    Amount of funding requested from the MES
  \end{enumerate}
 \item
  The AVPC shall review the budget proposals presented and suggest an
  amount to be awarded to each MES Club or Team. The MES Executives will
  review the suggested funding for each MES Club or Team and decide on
  the final amount that each club is to receive.
 \item
  All funding will be dispersed through approved expenses (see MES
  Bylaws Section H.2).
 \item
  If an MES Club or Team has not claimed any of their allotted funding
  by the beginning of second term, they must submit an activities report
  and progress update to the VPF by January 31st of the year, even if
  money is spent by the MES Club or Team during the month of January.

  \begin{enumerate}
   \item
    Once these documents are reviewed it is up to the VPF's discretion
    to either find the explanations satisfactory, or to arrange a
    meeting with the MES Club or Team to discuss their funding going
    forward.
   \item
    If these documents are not received, the MES Club or Team's funding
    will be reallocated to other MES Clubs and Teams.

  \end{enumerate}
\end{enumerate}

\hypertarget{conference-competition-funding-policy}{%
 \subsubsection{Conference / Competition Funding
  Policy}
 \label{conference-competition-funding-policy}}
\begin{enumerate}
 \item
  Open conferences and competitions

  \begin{enumerate}
   \item
    The MES will fund up to 50\% of the delegate, transportation, and
    accommodation fees for all delegates.
   \item
    Delegations requesting funding from the Open Conference and
    Competitions Fund must fill out the MES Conference/Competition
    Application form (Appendix N) and submit it to the VPX.
   \item
    The funding shall be approved:

    \begin{enumerate}
     \item
      By a vote from the MES Council if in an amount greater than \$500,
      in which case the head delegate from the group requesting funding
      must present to the MES Council to receive the funding.
     \item
      At the discretion of the VPX if in an amount less than \$500, in
      which case the VPX must update the MES Council on what funding was
      approved.
    \end{enumerate}
   \item
    In the event that a presentation needs to be made, the presentation
    shall be made before attending the conference or competition and
    shall be evaluated based on the following criteria:

    \begin{enumerate}
     \item
      Delegate/entrance fees
     \item
      Location of the event
     \item
      Number of MES members to attend
     \item
      Length and reputation of the event
     \item
      The degree of effort put into fundraising by the delegates
     \item
      Benefits for the MES
     \item
      History and value of the conference or competition
    \end{enumerate}
   \item
    The head delegate must also submit a report to the VPX, to be
    published in The Frequency and/or the MES website no later than one
    month after the conference or competition has concluded.

    \begin{enumerate}
     \item
      The report shall include updates on activities of the delegates,
      tangible outcomes, and effectiveness of the delegation in
      representing McMaster University.
    \end{enumerate}
   \item
    The following funding mechanism will be used once the conference or
    competition has been approved for funding:

    \begin{itemize}
     \item
      Delegation is between 1-20 MES members -- 50\% of the delegate fees,
      transportation and accommodation up to a maximum for \$100 per
      delegate and 1 head delegate to a maximum of \$150, not exceeding
      \$1000 per delegation
     \item
      Delegation is between 20-50 MES members -- 50\% of delegate fees,
      transportation, and accommodation up to a maximum of \$75 per delegate
      and 2 head delegates to a maximum of \$150 each, not exceeding \$1500
      per delegation
     \item
      Delegation is above 50 MES members -- 50\% of delegate fees,
      transportation \& accommodation up to a maximum of \$50 per delegate
      and 2 head delegates to a maximum of \$150 each, not exceeding \$2000
      per delegation.
    \end{itemize}

   \item
    The delegation must submit the signed Delegate Code of Conduct
    (Appendix F) prior to attending the conference. Failing to do so
    will give the MES Executive the right to revoke reimbursement.
   \item
    MES funding approval (while not preferred) can be given post-event
    on the condition that the MES Executive has been made aware (in
    writing) of the event and that MES members are attending it. The
    amount given will still be determined by the above criteria.
   \item
    If the MES Council approves funding, the group must submit receipts
    for all expenses being covered by the MES no later than one month
    after the conference or competition has concluded.
   \item
    Inappropriate or unprofessional delegate behavior gives the MES
    Executives the right to revoke reimbursement.
   \item
    The funding pool is limited and may be exhausted before all
    delegations apply. The VPF shall decide whether requests are
    acceptable and within the limits of the MES Budget.
  \end{enumerate}

 \item
  Affiliate conferences
  \begin{enumerate}
   \item
    The MES will fund 100\% of the delegate, transportation, and
    accommodation fees for the VPX to attend all conferences that have a
    plenary

    \begin{enumerate}
     \item
      This funding is conditional and may be revoked if the VPX fails to
      give a presentation to the MES Council outlining the outcomes of
      the conference.
     \item
      The MES will fund 100\% of the delegate, transportation, and
      accommodation fees for up to two representatives (ideally the
      President and the VPX) to attend ESSCO Presidents' Meeting
    \end{enumerate}
   \item
    The MES will fund at least 50\% of the delegate, transportation, and
    accommodation fees for any other delegates requesting financial aid
    chosen by the Delegate Selection Committee.

    \begin{enumerate}
     \item
      All delegates must write an article or organize an event or
      activity, up to the discretion of the VPX.
     \item
      Delegates approved by the Delegate Selection Committee as a
      conference attendee but not for MES funding are able to attend the
      affiliate conference without any reimbursement of funds.
     \item
      It should be noted that if any delegate has already received
      funding for another affiliate conference in the same academic
      year, that that delegate can only be granted funding for an
      affiliate conference if the reimbursement rate remain consistent
      across all delegates other than the VPX and President for that
      same conference.
    \end{enumerate}
   \item
    The head delegate of the delegation receiving funding must fill out
    the MES Conference/Competition Application form (Appendix N) and
    submit it to the VPA.
   \item
    The delegation must submit the signed Delegate Code of Conduct
    (Appendix F) prior to attending the conference. Failing to do so
    will give the MES Executive the right to revoke reimbursement.
  \end{enumerate}
 \item
  Competitions

  \begin{enumerate}
   \item
    The MES will fund the registration and transportation fees for MEC
    winners to attend the Ontario Engineering Competition, and Ontario
    Engineering Competition winners to attend the Canadian Engineering
    Competition, as organized by the McMaster Engineering Competition
    Chair(s).

    \begin{enumerate}
     \item
      Winners who are MES members will receive up to 100\% funding while
      non-MES member winners will only receive up to 50\% funding.
    \end{enumerate}
   \item
    The McMaster Engineering Competition Chair(s) shall write an article
    for The Frequency and/or the MES Website to be published no later
    than two weeks after each competition, detailing the results of the
    competitions.
   \item
    The head delegate of the delegation receiving funding must fill out
    the MES Conference/Competition Application form (Appendix N) and
    submit it to the VPA.
   \item
    The delegation must submit the signed Delegate Code of Conduct
    (Appendix F) prior to attending the conference. Failing to do so
    will give the MES Executive the right to revoke reimbursement.

  \end{enumerate}
\end{enumerate}

\hypertarget{intramural-funding-policy}{%
 \subsubsection{Intramural Funding
  Policy}
 \label{intramural-funding-policy}}
\begin{enumerate}
 \item
  The MES will sponsor intramural sports teams composed of at least 80\%
  MES members. These teams shall be reimbursed up to the percentage of
  MES members on their roster. If the intramural budget is
  underperforming the VPF and Sports Coordinator(s) reserve the
  authority to lower the mandatory member composition to a minimum of
  75\% MES members. If this happens there would be retroactive
  reimbursements to teams from the first semester that had already
  submitted their funding documents.
 \item
  The MES will fund the cost of registration up to \$300, matching the
  percentage of MES members on the team, not including any required
  deposits.
 \item
  To receive funding, a team must submit the following items to the
  Sports Coordinator(s):

  \begin{enumerate}
   \item
    A completed copy of Appendix X -- Intramurals Funding, including a
    full list of the team's playoff roster, including programs of study
    and student numbers of each team member, signed by each member of
    the team, and indicating the percentage of MES members.
   \item
    A standard expense report indicating the value of funding which
    appropriately reflects the percentage of MES members and the
    deduction of any required deposits.
   \item
    Copy of the receipt as proof of registration.
  \end{enumerate}
 \item
  The Sports Coordinator(s) shall review requests for funding to ensure
  they meet all requirements, and forward expense reports and receipts
  to the VPF.
 \item
  The Sports Coordinator(s) will be notified of the budget by the VPF
  once the budget is approved. The Sports Coordinator(s) should only
  approve requests within the limits of the budget.
 \item
  The funding pool is limited and may be exhausted before all teams
  apply. The VPF shall decide whether requests are acceptable and within
  the limits of the MES Budget.
\end{enumerate}

\hypertarget{special-projects-funding-policy}{%
 \subsubsection{Special Projects Funding
  Policy}
 \label{special-projects-funding-policy}}
\begin{enumerate}
 \item
  MES members requesting funding for a special project must:

  \begin{enumerate}
   \item
    For funding requests less than \$500:

    \begin{enumerate}
     \item
      Submit the MES Special Projects Application form, created and
      maintained by the Special Projects Coordinator(s)
     \item
      Receive a majority vote from the VPSL, the VPF and the President
    \end{enumerate}
   \item
    For funding requests between \$500 and \$1000:

    \begin{enumerate}
     \item
      Contact the Administrator such that they will be on the agenda for
      an MES Council meeting.
     \item
      Submit a motion to be put forth at an MES Council meeting, stating
      the funding amount being requested and the project for which the
      funds are being requested.
     \item
      Present to the MES Council the purpose of the project, details of
      the funds requested, and the proposed benefit to students.
    \end{enumerate}
   \item
    For funding requests greater than \$1000:

    \begin{enumerate}
     \item
      Contact the Administrator such that they will be on the agenda for
      an MES General meeting.
     \item
      Submit a motion to be put forth at an MES General meeting, stating
      the funding amount being requested and the project for which the
      funds are being requested.
     \item
      Present to the MES Members the purpose of the project, details of
      the funds requested, and the proposed benefit to students
    \end{enumerate}
  \end{enumerate}
 \item
  MES Groups and Teams are not eligible for this funding.
 \item
  The funding pool is limited and may be exhausted before any
  application. The VPF shall decide whether requests are acceptable and
  within the limits of the MES Budget.

\end{enumerate}

\hypertarget{accounts-and-cash}{%
 \subsection{Accounts and Cash}
 \label{accounts-and-cash}}

\hypertarget{accounts}{%
 \subsubsection{Accounts}
 \label{accounts}}
\begin{enumerate}
 \item
  The MES shall be the owner of at least:

  \begin{enumerate}
   \item
    One account at an outside chartered bank or trust company
   \item
    One McMaster University account.
  \end{enumerate}
 \item
  The chartered bank account shall be used for the majority of daily
  operations such as deposits, purchases, and payment of bills.
 \item
  The VPF will administer the chartered bank account, which shall be a
  corporate account requiring three authorized signing officers, with
  two out of the three signatures required on each cheque. The three
  signing officers shall be the President, VPF and VPSL of the MES.
 \item
  The VPF is responsible for administering all cheques, transfers, and
  deposits, and must receive approval from the proper levels of the MES
  Council for all payments that have not been previously granted
  approval by way of the budget.

\end{enumerate}

\hypertarget{investments}{%
 \subsubsection{Investments}
 \label{investments}}
\begin{enumerate}
 \item
  Funds held by the MES during the school year, in particular in the
  period between receipt of student fees in October and payment of
  Engineering Co-op and Career Services (ECCS) fees in March, may be
  invested by the MES Executive in a higher return venture.
 \item
  The investment must be agreed upon unanimously by the MES Executive,
  and must involve no risk of loss of funds.
 \item
  The VPF and one of the other signing officers will carry out the
  investment.
 \item
  Investments will be handled through the chartered bank where the
  current MES external account is held, unless the MES Executive deems
  it necessary to use an alternate chartered bank or trust company.

\end{enumerate}

\hypertarget{reserve-funds}{%
 \subsubsection{Reserve Funds}
 \label{reserve-funds}}
\begin{enumerate}
 \item
  The MES shall maintain a flexible (no risk) investment account, the
  sum of which shall constitute a reserve surplus fund (ECCS Fees held
  by the MES do not count towards this surplus, (see MES Bylaws Section
  H.4.2).
 \item
  Funds from this reserve shall be used at the discretion of the MES
  Executive as per Funding Policy (see MES Bylaws Section H.3).
 \item
  MES members requesting funding from Reserve Funds must submit a
  proposal to the MES Executive.
 \item
  This fund shall not be used to finance any of the operations mentioned
  within the operating budget of the fiscal year. (Such as, but not
  limited to -- Social Events, Groups \& Teams, Conferences, etc.)
 \item
  For all approved projects, the individual/group must submit receipts
  for all expenses being covered by the MES prior to receiving
  reimbursement. The individual/group must also meet all requirements
  detailed in the Sponsorship Checklist (see Appendix AA) before
  receiving reimbursement.

\end{enumerate}

\hypertarget{contingency}{%
 \subsubsection{Contingency}
 \label{contingency}}
\begin{enumerate}
 \item
  MES recognizes that prudent management of the society's resources
  require that the following funds be set aside in order to protect the
  MES in times of adverse economic condition or where the need for major
  expenditure may arise:

  \begin{enumerate}
   \item
    The Operating Contingency budget shall be set to \$30,000. The VPF
    shall allocate some of the Operating Contingency funds towards other
    operating budgets, as the year goes on and the operating finance
    risks are low.
   \item
    The Financial Contingency budget shall be set to \$70,000. Any
    expenditure from this budget shall be only done at the discretion of
    the President and the VPF, and should be replenished as soon as
    possible. This budget shall be treated as Mandatory Retained
    Earnings, and should be held separate from the MES Operating Budget.

  \end{enumerate}
\end{enumerate}

\hypertarget{petty-cash}{%
 \subsubsection{Petty Cash}
 \label{petty-cash}}
\begin{enumerate}
 \item
  Petty cash may only be held by the VPF to facilitate the sale of
  tickets to social events.
 \item
  Petty cash will be held using the cash box or the safe in the MES
  Office. The cash box must be kept in a secure location at all times.
 \item
  Petty cash in the safe should not exceed \$5000.
 \item
  Large amounts of money that are to be deposited in the bank may be
  kept in the safe by the VPF for a short period of time pending the
  next bank deposit.
 \item
  The VPF should empty the safe at least once a month.
 \item
  The combination lock for the safe shall be changed once a year
  immediately after the VPF has been ratified into the position. Only
  the Vice President Finance and President shall know the combination.

\end{enumerate}

\hypertarget{student-fees}{%
 \subsection{Student Fees}
 \label{student-fees}}
\begin{enumerate}
 \item
  MES student fees are collected yearly with tuition from each
  undergraduate engineering student by McMaster Financial Services.
 \item
  The per student MES student fee must be represented explicitly in the
  budget posted to the MES website upon approval by the MES council.
 \item
  All of the funds created by these fees are turned over to the MES by
  Financial Services in October by way of a cheque. This cheque is to be
  deposited in the external account such that the money can be used to
  cover operating expenses during the school year, with the excess being
  invested in a low-risk venture through the bank.
 \item
  An additional \$50 voluntary contribution will be collected with MES
  fees from each undergraduate engineering student for the McMaster
  Laboratory Advancement Benefaction Endowment Fund.
 \item
  An additional \$50 fee will be collected with tuition from each
  undergraduate engineering student to pay off the construction of the
  Gerald Hatch Centre.
 \item
  Fees shall be increased by the Consumer Price Index upon a majority
  vote by the MES Executive.

\end{enumerate}

\hypertarget{accounting}{%
 \subsection{Accounting}
 \label{accounting}}
\begin{enumerate}
 \item
  The VPF shall keep comprehensive records of all transactions through
  the MES accounts, by way of books that are to be kept in accordance
  with good accounting practice. MES shall employ a bookkeeper at the
  discretion of the VPF.
 \item
  Records may be kept through use of personal computer accounting
  software, which is to be available on the MES office computer.
 \item
  The VPF shall have the books balanced at the end of each semester and
  prepared for on demand perusal.
 \item
  The VPF reserves the right to transfer fees to a student account
  should a cheque not clear due to insufficient funds on part of the
  MES.

\end{enumerate}

\hypertarget{endowment-funds}{%
 \subsection{Endowment Funds}
 \label{endowment-funds}}
\hypertarget{the-mcmaster-laboratory-advancement-benefaction-endowment-fund-i.e.-maclab}{%
 \subsubsection{The McMaster Laboratory Advancement Benefaction
  Endowment Fund (i.e.
  macLAB)}
 \label{the-mcmaster-laboratory-advancement-benefaction-endowment-fund-i.e.-maclab}}
\begin{enumerate}
 \item
  The administration of the McMaster Laboratory Advancement Benefaction
  Endowment Fund is the responsibility of the macLAB Board of Directors.
 \item
  The macLAB Board of Directors will administer the fund in accordance
  with McMaster Laboratory Advancement Benefit Endowment Fund Bylaws.
 \item
  The President, VPF, and the VPA of the MES shall sit on the Board of
  Directors.
 \item
  The macLAB Board of Directors will administer an opt-out opportunity
  for all undergraduate engineering students by the end of December each
  year, and in accordance with the McMaster Laboratory Advancement
  Benefaction Endowment Fund Bylaws.
 \item
  The McMaster Laboratory Advancement Benefit Endowment FundBylaws may
  be changed and updated without approval from the MES Council.
 \item
  Refer to the McMaster Laboratory Advancement Benefaction Endowment
  Fund Bylaws for more information.

\end{enumerate}

\hypertarget{honoraria-awards}{%
 \subsection{Honoraria \& Awards}
 \label{honoraria-awards}}
\begin{enumerate}
 \item
  The MES Executive may choose to honor any MES member by way of an
  award or gift.
 \item
  The purchase of these awards or gifts must have direct approval from
  the MES Council before the purchase is made.

\end{enumerate}

\hypertarget{cash-advance}{%
 \subsection{Cash Advance}
 \label{cash-advance}}

\begin{enumerate}
 \item
  Cash Advances may be provided to MES affiliates, Clubs, Groups,
  Teams, or event organizers. The following criteria must be met:

  \begin{enumerate}
   \item
    The borrower must be an MES member.
   \item
    The borrower has completed the Cash Advance Contract with the VPF.
   \item
    The borrower pays back the advance by the agreed upon due date.
   \item
    The cash advance must be approved by the President and VPSL.
  \end{enumerate}
\end{enumerate}

\hypertarget{donations}{%
 \subsection{Donations}
 \label{donations}}
\begin{enumerate}
 \item
  All donations made by the MES shall fall under normal financial
  policies for release of funds, based on dollar amount, except in the
  case where the money is raised by fundraising techniques.
 \item
  Proof of donation must be provided to the VPF for records.
 \item
  All donations made by the MES must be publicly reported.
\end{enumerate}

% \end{document}