% !TEX program = xelatex

% \documentclass{article}
% \usepackage[letterpaper, portrait, margin=1in]{geometry}
% \usepackage{hyperref}
% \usepackage{alphalph}
% \usepackage{tabularray}
% \usepackage{longtable}
% \usepackage{booktabs}
% \usepackage{color}

% \setcounter{secnumdepth}{5}
% \renewcommand{\labelenumi}{\alphalph{\value{enumi}})}
% \renewcommand{\labelenumii}{\alph{enumii}.}
% \renewcommand{\labelenumiii}{\roman{enumiii}.}

% \begin{document}

\hypertarget{committees}{%
 \section{\texorpdfstring{\emph{COMMITTEES}}{COMMITTEES}}
 \label{committees}}

\hypertarget{general}{%
 \subsection{General}
 \label{general}}
Committees shall operate by majority vote procedure. The Chair shall
have no voting rights at the meeting except in the case of a tie. The
Chair has the deciding vote in the case of a tie.

\hypertarget{responsibilities-of-the-chair}{%
 \subsubsection{Responsibilities of the
  Chair}
 \label{responsibilities-of-the-chair}}

\begin{enumerate}
 \item
  The Chair will conduct all meetings of their committee to ensure each
  member has time to share their views and ideas with other members of
  the committee.
 \item
  The Chair shall ensure that the meetings will maintain a sense of open
  friendliness, and correct the behaviour of certain members should they
  be promoting ill will, or feelings of discomfort, towards other
  members of the committee.
 \item
  The Chair has the right to discuss removal of a committee member with
  the President should they decide the committee cannot function
  properly with this member present. The President will decide whether
  removal seems necessary, and instruct the Chairperson accordingly.
\end{enumerate}

\hypertarget{membership}{%
 \subsubsection{Membership}
 \label{membership}}
\begin{enumerate}
 \item
  Each MES Council member shall serve on at least two active committees
  - either Standing or Special.
 \item
  The Chairs and Council membership for each Standing Committee shall be
  ratified, and made public, by the second MES meeting following the
  start of each school year.
 \item
  The Chair is responsible to recruit members for their committee from
  the general membership of the MES.

\end{enumerate}

\hypertarget{welcome-week-orientation-planning-committee}{%
 \subsection{\texorpdfstring{Welcome Week Orientation Planning
   Committee
  }{Welcome Week Orientation Planning Committee }}
 \label{welcome-week-orientation-planning-committee}}
\begin{enumerate}
 \item
  The Co-Orientation Coordinators shall select the members of the
  Welcome Week Orientation Planning Committee, using an application
  process and interviewing if necessary.
 \item
  The Welcome Week Orientation Planning Committee members must have
  served as an engineering Welcome Week Representative for at least one
  year, as well as be familiar both with the MES and with the
  Engineering Welcome Week,
 \item
  Candidates are not required to hold an MES membership before applying
  for a position, however; individuals must become MES members once
  selected for this role.

  \begin{enumerate}
   \item
    Individuals may hold a position on the Welcome Week Orientation
    Planning Committee without holding an MES membership. This exception
    shall be granted on a financial need basis at the discretion of the
    Chief Returning Officer.
  \end{enumerate}
 \item
  The purpose of the Welcome Week Orientation Planning Committee is to
  help the Co-OCs with selecting Welcome Week Representatives and
  organizing Welcome Week.
 \item
  The Welcome Week Orientation Planning Committee may continue to assist
  the transition of first year students throughout the year.
 \item
  The Co-OCs and the Welcome Week Orientation Planning Committee are
  responsible for organizing all events related to Welcome Week
  including:

  \begin{enumerate}
   \item
    Summer communication to Welcome Week Reps indicating details of the
    summer bonding activity (Catapult), a schedule for Welcome Week and
    any other relevant information.
   \item
    Summer mailing to first year students indicating schedule and
    description of Welcome Week, MES first year flyer, and other
    relevant information.
   \item
    All events to take place during Welcome Week.
  \end{enumerate}
 \item
  The Welcome Week Orientation Committee will ensure that all
  activities, as well as the behaviour of the Welcome Week Reps, fall
  within the Welcome Week Values and Guiding Principles and the
  guidelines of the Student Success Centre.
 \item
  The Welcome Week Orientation Committee will prepare a budget for
  Welcome Week activities to be presented at the first council meeting
  before the summer.
 \item
  The President and the McMaster Faculty of Engineering's Assistant Dean
  (Studies) shall approve the final list of Welcome Week Representatives
  after the selection process and the final list of Welcome Week
  Representative names after the naming process.

\end{enumerate}

\hypertarget{ad-hoc-committees}{%
 \subsection{\texorpdfstring{Ad Hoc Committees
  }{Ad Hoc Committees }}
 \label{ad-hoc-committees}}
\begin{enumerate}
 \item
  Ad Hoc Committees will be established as seen fit by the MES
  Executive, for the purposes of accomplishing short-term goals of the
  MES.
 \item
  The MES Executive shall appoint Chairs for Ad Hoc Committees.
 \item
  The MES Executive will determine the purpose of these Committees in
  conjunction with the approved Chair for the Committee.
 \item
  The MES Executive will determine the duration of an Ad Hoc Committee's
  existence in conjunction with the Chair of that Committee.

\end{enumerate}

\hypertarget{standing-committees}{%
 \subsection{Standing Committees}
 \label{standing-committees}}
Standing Committees shall hold meetings at least once a term for the
entire academic year. The Chair may call as many additional meetings
each month as they deems necessary.

\hypertarget{advertising-committee}{%
 \subsubsection{Advertising Committee}
 \label{advertising-committee}}

\begin{enumerate}
 \item
  The VPC shall chair the Advertising Committee.
 \item
  The Advertising Committee is responsible for ensuring advertising
  campaigns are made for each event coordinated by an event coordinator,
  MES Executive, or AVP, and sponsored by the MES.
 \item
  Promote external campaigns on MES Social Media platforms where
  appropriate, including but not restricted to sharing program events,
  external opportunities, and ECCS initiatives.
 \item
  The members of the Advertising Committee shall consist of:

  \begin{enumerate}
   \item
    Graphic Designers
   \item
    Social Media Coordinator(s)
   \item
    Website Coordinator(s)
   \item
    Publications Editor
   \item
    Photographer/Videographer(s)
  \end{enumerate}
 \item
  The Committee is responsible for that media content (such as flyers,
  posters, posts, stories, and videos) are created, in coordination with
  the brand, for all MES events.

\end{enumerate}

\hypertarget{awards-selection-committee}{%
 \subsubsection{Awards Selection
  Committee}
 \label{awards-selection-committee}}
\begin{enumerate}
 \item
  The Chief Returning Officer shall chair the committee. The Chair is
  responsible for finding a Faculty member and a student to sit on the
  Committee.
 \item
  The members of the Awards Selection Committee shall consist of:

  \begin{enumerate}
   \item
    Administrator
   \item
    Vice President, Student Life
   \item
    One Faculty member
   \item
    One student who is not a MES Council member
  \end{enumerate}
 \item
  The Awards Selection Committee is responsible for administering
  procedures for the following awards:

  \begin{enumerate}
   \item
    ``Image of an Engineer'' Awards
   \item
    Faculty Appreciation Award
   \item
    Outstanding Teaching Assistant Award
  \end{enumerate}
 \item
  The Awards Selection Committee will investigate and publicize other
  potential award programs open to undergraduate engineering students.

\end{enumerate}

\hypertarget{club-leaders-committee}{%
 \subsubsection{Club Leaders
  Committee}
 \label{club-leaders-committee}}
\begin{enumerate}
 \item
  The AVPC shall chair the Club Leaders Committee.
 \item
  The members of the Club Leaders committee shall consist of:

  \begin{enumerate}
   \item
    Vice President, Student Life
   \item
    Program Representatives
   \item
    First Year Representatives
   \item
    Club Presidents
   \item
    Associate Vice President, Clubs
  \end{enumerate}
\end{enumerate}

The purpose of the Club Leaders Committee is to discuss relevant issues,
coordinate club funding with the help of the AVPC, and facilitate
interaction between clubs.

\hypertarget{delegate-selection-committee}{%
 \subsubsection{Delegate Selection
  Committee}
 \label{delegate-selection-committee}}

\begin{enumerate}
 \item
  The VPX shall chair the Delegate Selection Committee.
 \item
  The Delegate Selection Committee shall accept applications from
  McMaster undergraduate engineering students interested in becoming a
  delegate on behalf of the MES at closed conferences.
 \item
  The members of the Delegate Selection Committee shall be a minimum of
  3 individuals consisting of:

  \begin{enumerate}
   \item
    Vice President, External Relations
   \item
    The remainder of the committee shall be composed of other faculty of
    engineering students not attending the conference, as selected by
    the VPX
  \end{enumerate}
 \item
  The Delegate Selection Committee should remain uniform when selecting
  delegates unless a conflict arises from the membership indicated in
  c), above.
\end{enumerate}

\hypertarget{editorial-review-committee}{%
 \subsubsection{Editorial Review
  Committee}
 \label{editorial-review-committee}}
\begin{enumerate}
 \item
  The Publications Editor shall chair the Editorial Review Committee.
 \item
  The Editorial Review Committee must review and unanimously approve of
  all material being considered for all MES publications in accordance
  with the editorial policies of the MES (see ``Services'', Section
  F.1.3).
 \item
  The members of the Editorial Review Committee shall consist of:

  \begin{enumerate}
   \item
    Editors of the Plumbline, Frequency and Handbook.
   \item
    Equity and Inclusion Officer.
   \item
    Two members of the MES Executive.
   \item
    Two McMaster undergraduate students in the Faculty of Engineering.
  \end{enumerate}
 \item
  The Editorial Review Committee shall assist the Publications Editor
  with their responsibilities.
 \item
  The Editorial Review Committee shall be responsible for the
  distribution of MES productions.

\end{enumerate}

\hypertarget{elections-committee}{%
 \subsubsection{Elections Committee}
 \label{elections-committee}}
\begin{enumerate}
 \item
  The Chief Returning Officer shall chair the Elections Committee.
 \item
  The voting members of the Elections Committee shall consist of:

  \begin{enumerate}
   \item
    President
   \item
    Vice President, Student Life
   \item
    Equity and Inclusion Officer
  \end{enumerate}
 \item
  The Elections Committee shall fulfill their responsibilities as set
  out in the MES Bylaws Section C.1.7.

\end{enumerate}

\hypertarget{maclab-board-of-directors}{%
 \subsubsection{macLAB Board of
  Directors}
 \label{maclab-board-of-directors}}
See the McMaster Laboratory Advancement Benefaction Endowment Fund
By-Laws document for details.

\hypertarget{first-year-committee}{%
 \subsubsection{First Year Committee}
 \label{first-year-committee}}

\begin{enumerate}
 \item
  The VPSL shall chair the First Year Committee
 \item
  The members of the First Year Committee shall consist of:

  \begin{enumerate}
   \item
    Six First Year Representatives (see MES Bylaws Section B.4.3)
  \end{enumerate}
 \item
  The purpose of the First Year Committee is to coordinate activities
  and fundraising initiatives for the first year class.
\end{enumerate}

\hypertarget{kipling-committee}{%
 \subsubsection{Kipling Committee}
 \label{kipling-committee}}
\begin{enumerate}
 \item
  The Kipling Coordinator(s) shall chair the Kipling Committee.
 \item
  The members of the Kipling Committee shall consist of:

  \begin{enumerate}
   \item
    Associate Vice President, Events
   \item
    Vice President, Student Life
  \end{enumerate}
 \item
  The Kipling Committee is responsible for organizing the Kipling Dinner
  to follow the Iron Ring Ceremony.
 \item
  The Kipling Committee shall establish a "Clean-up" Committee for the
  purpose of showing good faith to the University by helping to clean up
  unapproved Kipling ``pranks'' known to occur the night before the Iron
  Ring Ceremony.

\end{enumerate}

\hypertarget{newsletter-committee}{%
 \subsubsection{Newsletter Committee}
 \label{newsletter-committee}}
\begin{enumerate}
 \item
  The Frequency Editor(s) shall chair the Newsletter Committee.
 \item
  The members of the Newsletter Committee shall consist of:

  \begin{enumerate}
   \item
    Sports Coordinator(s)
   \item
    Administrator
   \item
    Vice President,Student Life
   \item
    Associate Vice President, Events
   \item
    Program Representatives
   \item
    Information Technology Coordinator(s)
   \item
    Any other undergraduate engineering students
  \end{enumerate}
 \item
  The Newsletter Committee's purpose is to provide relevant articles for
  the production of the Frequency to keep engineering students up to
  date and informed on MES activities.

\end{enumerate}

\hypertarget{social-committee}{%
 \subsubsection{Social Committee}
 \label{social-committee}}
\begin{enumerate}
 \item
  The AVPE shall chair the Social Committee,
 \item
  The members of the Social Committee shall consist of:

  \begin{enumerate}
   \item
    Fireball Coordinator(s)
   \item
    Vice President, Student Life
   \item
    Kipling Coordinator(s)
   \item
    Wellness Coordinator(s)
   \item
    Culture Chair(s)
   \item
    Outgoing Associate Vice President, Events
   \item
    Any other undergraduate engineering students
  \end{enumerate}
 \item
  The remainder of the Social Committee shall be composed of a variety
  of faculty of engineering students from different years and
  disciplines.
 \item
  Any Social Committee member who volunteers for a particular event
  shall receive a discounted or free ticket to the event at the
  discretion of the Associate Vice President Events. \hspace{0pt}Tickets
  for Kipling Formal and Fireball will be available at a reduced cost
  for volunteers (at the discretion of the respective coordinator(s)).
 \item
  The Social Committee shall be responsible for the coordination of the
  following specific events:

  \begin{enumerate}
   \item
    Engineering Pubs
   \item
    Fireball
   \item
    Kipling

  \end{enumerate}
\end{enumerate}

\hypertarget{sports-committee}{%
 \subsubsection{Sports Committee}
 \label{sports-committee}}
\begin{enumerate}
 \item
  The Sports Coordinator(s) shall chair the Sports Committee.
 \item
  The members of the sports committee shall consist of:

  \begin{enumerate}
   \item
    Associate Vice President, Events
   \item
    Vice President, Student Life
   \item
    Program Representatives
   \item
    First Year Representatives
  \end{enumerate}
 \item
  The Sports Committee is responsible for organizing intramural sporting
  activities.
 \item
  The Sports Committee shall organize any special sporting events which
  would be of interest to the MES such as:

  \begin{enumerate}
   \item
    Intrafaculty challenges
   \item
    Interfaculty challenges
   \item
    Ratboy Memorial Soccer Tournament
   \item
    Dodgeball Tournament

  \end{enumerate}
\end{enumerate}

\hypertarget{mcmaster-engineering-competition-committee}{%
 \subsubsection{McMaster Engineering Competition
  Committee}
 \label{mcmaster-engineering-competition-committee}}
\begin{enumerate}
 \item
  The MEC Chair(s) shall chair and select the members of the MEC
  Committee.
 \item
  The Committee is responsible for ensuring the success of MEC as a
  qualifying event for OEC.
 \item
  The structure of the Committee shall be determined by the MEC Chair(s)
  and is to be approved by the VP Academic.

\end{enumerate}

\hypertarget{culture-committee}{%
 \subsubsection{Culture Committee}
 \label{culture-committee}}
\begin{enumerate}
 \item
  The Culture Chair(s) shall chair the Culture Committee.
 \item
  Any other undergraduate engineering student can serve on this
  committee.
 \item
  The Culture Committee is responsible for assisting the Culture Chair
  in the running of Santa Hog, the Santa Claus Parade, Pi Day, and any
  other events organized by the Culture Chair.
 \item
  The Culture Committee shall assist in the gathering and recording of
  information regarding tradition including but not limited to:

  \begin{enumerate}
   \item
    Songs/Cheers
   \item
    Redsuit / MES position information
   \item
    Contact information with past engineering graduates
   \item
    Stories from iii)

  \end{enumerate}
\end{enumerate}

\hypertarget{professional-development-committee}{%
 \subsubsection{Professional Development
  Committee}
 \label{professional-development-committee}}
\begin{enumerate}
 \item
  The Professional Development Coordinator(s) shall chair the
  Professional Development Committee.
 \item
  The Committee is responsible for assisting in the running of events
  such as:

  \begin{enumerate}
   \item
    LinkedIn photoshoots
   \item
    Professional development workshops
   \item
    Technical tutorials
   \item
    Any other initiatives from the Professional Development Committee
    Chair(s)
  \end{enumerate}
\end{enumerate}

\hypertarget{leadership-development-conference-committee}{%
 \subsubsection{Leadership Development Conference
  Committee}
 \label{leadership-development-conference-committee}}
\begin{enumerate}
 \item
  The Leadership Development Conference Coordinator(s) shall chair the
  Leadership Development Conference Committee
 \item
  The Committee is responsible for assisting in the running of the
  conference with positions such as:

  \begin{enumerate}
   \item
    VP Marketing
   \item
    VP Logistics
   \item
    VP Seminars
   \item
    Any other positions as deemed necessary by the Leadership
    Development Conference Coordinator(s)

  \end{enumerate}
\end{enumerate}

\hypertarget{wellness-committee}{%
 \subsubsection{Wellness Committee}
 \label{wellness-committee}}
\begin{enumerate}
 \item
  The Wellness Coordinator(s) shall chair and select the members of the
  Wellness Committee.
 \item
  The Wellness Committee is responsible for ensuring an appropriate
  amount of events related to student wellness, education, and stress
  relief are run each term.
 \item
  The structure of the Wellness Committee shall be determined by the
  Wellness Coordinator(s) and is to be approved by the VPSL

\end{enumerate}

\hypertarget{sustainability-committee}{%
 \subsubsection{Sustainability
  Committee}
 \label{sustainability-committee}}
\begin{enumerate}
 \item
  The Sustainability Coordinator(s) shall chair and select the members
  of the Sustainability Committee.
 \item
  The Sustainability Committee is responsible for ensuring an
  appropriate amount of events related to sustainability and leadership
  in sustainability are run each term.
 \item
  The Sustainability Committee shall seek ways that students can engage
  with the McMaster Community on sustainability, and ways in which
  engineering students can support sustainability efforts on and off
  campus.
\end{enumerate}

\hypertarget{academic-services-committee}{%
 \subsubsection{Academic Services
  Committee}
 \label{academic-services-committee}}
\begin{enumerate}
 \item
  The AVPAR shall chair the Academic Services Committee.
 \item
  The Academic Services Committee will be composed of upper year student
  `course leads' chosen by the AVPAR:

  \begin{enumerate}
   \item
    Course leads will be assigned by the AVPAR to run sessions on a case
    by case basis.
   \item
    Course leads will be responsible for supplying the AVPAR with a
    review/study plan before running a session.
   \item
    Course leads will be expected to show up to their assigned session
    and deliver a review session on the respective material for the
    course.
   \item
    Course leads will be paid for the hours spent running the sessions
    at a rate higher than the minimum wage and determined by the AVPAR.
  \end{enumerate}
 \item
  The Academic Services Committee is responsible for assisting in
  running MES Help Sessions throughout the year, as well as assisting
  the AVPAR with administrative responsibilities where appropriate.

\end{enumerate}

\hypertarget{equity-committee}{%
 \subsubsection{Equity Committee}
 \label{equity-committee}}
\begin{enumerate}
 \item
  The Equity and Inclusion Officer shall chair the Equity Committee.
 \item
  The Equity Committee shall consist of representatives selected by the
  chairs of the following committees:

  \begin{enumerate}
   \item
    Welcome Week Planning Committee
   \item
    McMaster Engineering Competition Committee
   \item
    Advertising Committee
   \item
    Delegate Selection Committee
   \item
    Awards Selection Committee
   \item
    Social Committee
   \item
    Sports Committee
   \item
    Wellness Committee
   \item
    Any other committees at the discretion of the Equity and Inclusion
    Officer.
  \end{enumerate}
 \item
  The mandate of the Equity Committee is to ensure that equity and
  inclusion concerns are being actively worked on within the MES.

\end{enumerate}

\hypertarget{first-year-experiential-conference-fyec-committee}{%
 \subsubsection{First Year Experiential Conference (FYEC)
  Committee}
 \label{first-year-experiential-conference-fyec-committee}}
\begin{enumerate}
 \item
  The FYEC Chair(s) shall chair and select the members of the FYEC
  Committee
 \item
  The FYEC Committee is responsible for planning and organizing all
  functions of the First Year Experiential Conference.
 \item
  The structure of the FYEC Committee shall be determined by the FYEC
  Chair(s) and is to be approved by the VPSL

\end{enumerate}

\hypertarget{bus-monitor-committee}{%
 \subsubsection{Bus Monitor Committee}
 \label{bus-monitor-committee}}
\begin{enumerate}
 \item
  The Bus Monitor Lead shall chair the Bus Monitor Committee.
 \item
  The Bus Monitor Committee shall consist of the following members:

  \begin{enumerate}
   \item
    Bus Monitor Lead
   \item
    AVP Events
   \item
    VP Student Life
   \item
    All volunteer bus monitors
  \end{enumerate}
 \item
  Any Bus Monitor Committee member who volunteers as a bus monitor for a
  particular event shall receive a discounted or free ticket into the
  event at the discretion of the AVPEs.\hspace{0pt} \hspace{0pt}Tickets
  for Kipling Formal and Fireball will be available at a reduced cost
  for bus monitors (at the discretion of the respective coordinator(s)).
 \item
  Online and in-person training for bus monitors will be provided over
  the summer.
 \item
  Bus monitors will be provided with apparel that aid in identifying
  them.
 \item
  \hspace{0pt}All bus monitors will be required to complete specific
  training, as outlined in Appendix AD.
 \item
  Should someone wish to bus monitor for an event during the year
  without completing the required training, they will be paired with
  someone that has completed the training. This is at the discretion of
  the event organizer

\end{enumerate}

\hypertarget{i.d.e.a.-committee-inclusivity-and-diversity-in-engineering-alliance}{%
 \subsubsection{I.D.E.A. Committee (Inclusivity and Diversity in
  Engineering
  Alliance)}
 \label{i.d.e.a.-committee-inclusivity-and-diversity-in-engineering-alliance}}
\begin{enumerate}
 \setcounter{enumi}{3}
 \item
  The Equity and Inclusion Officer shall chair the I.D.E.A. Committee.
 \item
  I.D.E.A. Committee will ensure that all voices and perspectives in the
  MES are being heard.
 \item
  This committee will facilitate communication between the WIE Society,
  NSBE and EngiQueers.
\end{enumerate}

\hypertarget{meetings}{%
 \section{\texorpdfstring{\emph{MEETINGS}}{MEETINGS}}
 \label{meetings}}

\hypertarget{general-meetings-sagm}{%
 \subsection{General Meetings (SAGM)}
 \label{general-meetings-sagm}}
\begin{enumerate}
 \item
  There shall be at least one General Meeting per term.
 \item
  Quorum at General Meetings shall be 3\% of the total McMaster
  undergraduate students in the Faculty of Engineering.
 \item
  Quorum for each MES Program Society is ten persons per club. No one
  person can count towards the quorum of more than one MES Program
  Society.
 \item
  The Chief Returning Officer will chair the General Meetings. In the
  absence or declination of the Chief Returning Officer to chair, a vote
  shall be held to elect a chair with a two thirds majority vote.
 \item
  There shall be no proxy voting at General Meetings.
 \item
  Motions for the General Meeting are due one week prior to the meeting
  to allow for adequate advertising and review. It is at the discretion
  of the Chief Returning Officer to accept motions submitted after the
  deadline.
 \item
  The General Meeting agenda is to be posted at least 24 hours in
  advance of meeting.
 \item
  General Meetings shall be conducted in accordance with the MES version
  of Robert's Rules of Order (see ``MES Robert's Rules of Order'',
  Appendix I).

  \begin{enumerate}
   \item
    All McMaster undergraduate students in the Faculty of Engineering
    have the right to vote at MES General Meetings
  \end{enumerate}
\end{enumerate}

\hypertarget{mes-council-meetings}{%
 \subsection{MES Council Meetings}
 \label{mes-council-meetings}}

\hypertarget{general-1}{%
 \subsubsection{General}
 \label{general-1}}

\begin{enumerate}
 \item
  MES Council meetings shall be held once every two weeks for the
  entirety of both semesters.
 \item
  Quorum will consist of two-thirds of all MES Council voting positions
  as is outlined within the MES Constitution. Unfilled First Year
  Representative positions shall not count towards quorum until the
  First Year Representative Election has taken place.
 \item
  The Administrator must be notified of any motions and agenda items at
  least 48 hours before the meeting. Motions and agenda items not
  submitted within this time period may not be permitted at the meeting,
  subject to decision by the MES Council.
 \item
  The Administrator must forward all motions for funding to the VPF upon
  receiving the motion(s) in order to provide time for review.
 \item
  The Administrator is responsible for recording the minutes of all
  proceedings of the meeting. If the Administrator is unavailable for
  all or part of a meeting, another chosen/volunteered MES Council
  member will record minutes in the Administrator's absence.
 \item
  The Chief Returning Officer shall chair all MES Council meetings. The
  President will chair any MES Council meetings in their absence. Should
  both the Chief Returning Officer and President decline or are unable
  to chair, a vote will be held to elect a new chair requiring a two
  thirds majority.
 \item
  MES Council meetings shall be conducted in accordance with the MES
  version of Robert's Rules of Order (see ``MES Robert's Rules of
  Order'', Appendix I).
 \item
  The adoption of the agenda will be motioned at the start of council
  meetings. Any amendments will be brought forth as motions to amend the
  agenda during this motion.
 \item
  The Administrator tallies and records any voting on motions. A second
  counter must confirm the Administrator's count.
 \item
  At the end of each MES Council meeting, before the motion to adjourn,
  there shall be an opportunity for new business to be added to the
  agenda pending a two thirds majority vote in favor to add the new
  business. New business must be in compliance with the restrictions of
  the Policy Manual, theConstitution, and this document.
 \item
  All submitted motions must follow the format outlined in Appendix Z
 \item
  MES Council is permitted to use a consent agenda, where multiple items
  of discussion can be grouped into a single motion and vote.
 \item
  MES Council may pass motions by unanimous consent, also known as
  general consent, when no elected Council member objects to a motion.
 \item
  The Chair is not required to pass the chair if they need to present,
  but must do so in the event where they are unable to moderate
  discussion in a proper and unbiased manner.
  \begin{enumerate}
    \item
     The Temporary Chair must:
     \begin{enumerate}
      \item
       be an elected Council member.
      \item
       be elected by majority vote.
      \item
       not pass the chair except when returning it to the Chair.
     \end{enumerate}
    \item
    The Chair automatically reassumes their role upon the conclusion of
    their presentation and discussion.
  \end{enumerate}

\end{enumerate}

\hypertarget{attendance}{%
 \subsubsection{Attendance}
 \label{attendance}}
\begin{enumerate}
 \item
  Attendance at MES Council meetings is mandatory for all MES Council
  elected positions.
 \item
  Any absences must be communicated to the Chief Returning Officerand
  Administrator at least 24 hours in advance.
 \item
  Suitable excuses for missing meetings include:

  \begin{enumerate}
   \item
    Illness
   \item
    Classes
   \item
    Writing tests
   \item
    Representing the MES in an official capacity at another event
   \item
    Other emergencies (at the discretion of the Chief Returning Officer)
  \end{enumerate}
 \item
  If a council member is unable to attend they should appoint someone to
  be their proxy and seek the approval of the Chief Returning Officer
  (See MES Bylaws Section G.2.3)
 \item
  If an MES Council member misses more than two MES Council meetings in
  a single semester without an approved excuse, they shall be removed
  from the MES Council at the discretion of the Chief Returning Officer.
  The position shall then be open for election or appointment
  accordingly.
\end{enumerate}

\hypertarget{proxy-voting}{%
 \subsubsection{Proxy Voting}
 \label{proxy-voting}}
\begin{enumerate}
 \item
  A voting Council member may proxy their vote at an MES Council meeting
  to another voting Council member or any Full Member of the MES.
 \item
  Approval of the proxy must be received from the Chief Returning
  Officer, and notice sent to the Administrator no less than 24 hours
  before the start of the meeting in question. A request to proxy will
  not be accepted at the start of or during any meeting.
 \item
  The proxy will remain in force for the duration of the selected
  meeting only.
 \item
  Notice of the proxy must be included on the meeting agenda and
  announced at the beginning of the meeting.
 \item
  Although the person initiating the proxy can give their voting
  preference to the voter, the person receiving the proxy can vote as
  they please. MES Council members should take this into consideration
  when choosing a person to vote in their absence.
\end{enumerate}

\hypertarget{quorum}{%
 \subsubsection{Quorum}
 \label{quorum}}

\begin{enumerate}
 \item
  Quorum will consist of both:

  \begin{enumerate}
   \item
    60\% of voting Council members in attendance
   \item
    75\% of Full Quorum votes are present and binding

    \begin{enumerate}
     \item
      Full Quorum is the total number of elected council members. Full
      Quorum is normally 29 votes, unless voting positions are vacant.
     \item
      A vote is considered binding if it is held by the elected position
      or a binding proxy.
    \end{enumerate}
  \end{enumerate}
 \item
  Under no circumstance will a proxy be considered binding if the proxy
  is for the President, Vice-President, or an Associate-Vice-President.
 \item
  There are five classes of reasons for missing a Council meeting:

  \begin{enumerate}
   \item
    MES or Faculty representation absence (e.g. ESSCO/CFES Conference,
    MES Team external competition)
   \item
    Non-repeating, academic (e.g. midterms, industry night)
   \item
    Non-repeating, non-academic (e.g. concert, club event, illness)
   \item
    Repeating, academic (e.g. night classes)
   \item
    Repeating, non-academic (e.g. club meetings)
  \end{enumerate}
 \item
  In cases i) and ii), any proxy will be counted as binding.
 \item
  In cases iii) and iv), program representative proxies will be counted
  as binding only if the proxy is a member of their respective
  constituency.
 \item
  In case v), program representative proxies will be counted as binding
  only if the proxy is a member of their respective program society
  executive.
 \item
  In cases iii), iv) and v), first year representative proxies will be
  counted as binding only if the proxy is a member of their respective
  constituency.
 \item
  Unfilled positions shall not count towards quorum until the respective
  elections have taken place and the positions are filled.

\end{enumerate}

\hypertarget{committee-meetings}{%
 \subsection{Committee Meetings}
 \label{committee-meetings}}
\begin{enumerate}
 \item
  Committee Chair(s) shall be responsible for organizing and conducting
  regular meetings in an efficient and orderly manner (see MES Bylaws
  Section D).

\end{enumerate}

\hypertarget{supervisory-meetings}{%
 \subsection{Supervisory Meetings}
 \label{supervisory-meetings}}
\begin{enumerate}
 \item
  All MES Council members will meet with their respective supervisors to
  discuss their progress or any problems they might be having on a
  regular basis.
 \item
  Supervisors are to make themselves and their resources available to
  the people they are supervising whenever possible.
\end{enumerate}

\hypertarget{mes-roberts-rules-of-order}{%
 \section{\texorpdfstring{\emph{MES ROBERT'S RULES OF
     ORDER}}{MES ROBERT'S RULES OF ORDER}}
 \label{mes-roberts-rules-of-order}}
The following are meeting procedures to be followed in General Meetings
and MES Council meetings. Robert's Rules of Order is a strict, but
practical system for running meetings. The MES follows its own version
of Robert's Rules of Order as outlined below.

\hypertarget{eleven-fundamental-rules-of-procedure}{%
 \subsection{ELEVEN FUNDAMENTAL RULES OF
  PROCEDURE}
 \label{eleven-fundamental-rules-of-procedure}}

\begin{enumerate}
 \item
  Principles

  \begin{enumerate}
   \item
    The right of majority to decide
   \item
    The right of minority to be heard
   \item
    The right of individual members
   \item
    The right of absentees
  \end{enumerate}
 \item
  All voting members of the MES Council are equal and their rights are
  equal. These rights are:

  \begin{enumerate}
   \item
    To attend council meetings
   \item
    To speak
   \item
    To move motions
   \item
    To second motions
   \item
    To vote
   \item
    To concede their rights to others
  \end{enumerate}
 \item
  All non-voting members of the MES Council are equal and their rights
  are equal. These rights are:

  \begin{enumerate}
   \item
    To attend council meetings
   \item
    To speak
   \item
    To movemotions
   \item
    To second motions
   \item
    To concede their rights to others
  \end{enumerate}
 \item
  All McMaster undergraduate students in the Faculty of Engineering are
  equal and their rights are equal. These rights are:

  \begin{enumerate}
   \item
    To attend council meetings
   \item
    To speak
   \item
    To move motions
  \end{enumerate}
 \item
  The chair has the following powers:

  \begin{enumerate}
   \item
    To speak when clarification is necessary
   \item
    To determine the speaking order
   \item
    To count votes
   \item
    To interpret the MES Policy Manual, Bylaws, and Constitution
   \item
    To recognize members
   \item
    To decide what is in order
   \item
    To remove members from council dependent on a two thirds majority
    vote of council.
  \end{enumerate}
 \item
  The rights of MES supersede the rights of individual members: Should a
  conflict arise between the rights of the MES and the rights of an
  individual member, the rights of MES shall take precedence.
 \item
  Quorum must be present at MES Council meetings for business to be
  done. Quorum shall consist of half of all voting members of the MES
  Council as outlined by the MES Constitution, unless otherwise stated
  in the MES Policy Manual, Bylaws, or Constitution.
 \item
  It is the responsibility of every voting member of the MES Council to
  vote. If they do not vote, it shall be assumed that they are
  abstaining from the vote.
 \item
  One speaker at a time: Only one speaker recognized by the chair has
  the right to talk at any given moment.
 \item
  Personal remarks are always out of order, the chair maintains the
  right, by a two thirds majority, to remove members from meetings for
  failing to comply with this rule.
 \item
  If there is ever a dispute on a decision/interpretation the chair has
  made, any voting member of the MES Council may move to challenge the
  chair and their decision. This motion, which would require a seconder,
  would open up discussion to the floor about the
  decision/interpretation. Each member will be allowed to speak once to
  the decision/interpretation, closing with the chair defending their
  decision/interpretation. Following the chair's statement, there shall
  be a vote. If there is a two thirds majority in favor of overruling
  the chair's decision/interpretation, the chair will accept the
  decision as their own and move onwards or the chair will be required
  to relinquish the chair.

\end{enumerate}

\hypertarget{motions}{%
 \subsection{MOTIONS}
 \label{motions}}
There are two kinds of motions: Main Motions and Secondary Motions.

\hypertarget{main-motions}{%
 \subsubsection{Main Motions}
 \label{main-motions}}

A main motion is defined as a proposal that certain action be taken or
an opinion be expressed by the organization.

\begin{enumerate}
 \item
  All main motions must be submitted to the Administrator at least 24
  hours before the meeting at which it will be moved.
 \item
  A main motion brought toward the MES will be read to the MES Council
  by the chair.
 \item
  A main motion must be seconded in order to proceed to presentation
  and/or debate.
 \item
  Movers and seconders may each make a short presentation outlining the
  action.
 \item
  Debate is struck. The chair recognizes members who wish to speak by
  stating their names. In the case of multiple speakers, a cue will be
  established and decided by the chair.
 \item
  Debate should continue as long as members wish to discuss the question
  unless the chair has put the question to a vote or secondary motions
  have been adopted to either limit or close debate.
 \item
  When a main motion is put to question, the chair shall restate the
  motion. Voting members are instructed to raise their hand when those
  in favour, opposition, or abstention are called by the chair. Any
  singular vote may be noted by name, exclusively by request from the
  member to whom that vote belongs.
 \item
  Voting results are counted by the chair and Administrator, and the
  result of the main motion is announced to the MES Council by the
  chair.
 \item
  A main motion shall pass if the votes in favour are counted to be half
  plus one of the total votes, unless specifically stated otherwise in
  the MES Constitution, Bylaws, or Policy Manual.
 \item
  All motions must be in compliance with the MES Constitution, Bylaws,
  and Policy Manual as interpreted by the chair.

\end{enumerate}
\hypertarget{secondary-motions}{%
 \subsubsection{Secondary Motions}
 \label{secondary-motions}}
Secondary motions are motions that may be made while the main motion is
on the floor and before it has been decided. More than one motion can be
on the floor but only one main motion. All pending motions must relate
to the main motion on the floor, no new business may be introduced.

Secondary motions have rank among each other. They are arranged in a
specific order in which they must be considered and acted upon, some
motions taking precedence over others. The purpose is to avoid confusion
when they are applied to a main motion. A motion of higher rank can be
made at the time that a motion of lower rank is on the floor.

\hypertarget{the-ranks-of-secondary-motions}{%
 \subsubsection{THE RANKS OF SECONDARY MOTIONS}
 \label{the-ranks-of-secondary-motions}}

\begin{table}[h]
 \centering
 \begin{tblr}{cells={c},hlines={1pt, black}}
  CALL TO QUESTION                              \\
  LIMIT OR EXTEND LIMITS TO DEBATE OR QUESTIONS \\
  POSTPONE TO A SPECIFIC TIME                   \\
  COMMIT OR REFER                               \\
  AMEND                                         \\
  \textbf{MAIN MOTION}                          \\
 \end{tblr}
\end{table}

Rank can be symbolized by the rungs of a ladder. Those motions on the
lower rungs must yield to the motions on the rung or rungs above. For
instance call to question, which calls for an immediate vote, takes
precedence over all motions below it. All secondary motions listed in
the table above must be seconded and are debatable.

\hypertarget{what-do-i-say}{%
 \subsubsection{WHAT DO I SAY?}
 \label{what-do-i-say}}

Secondary Motions Arranged From Lowest to Highest Rank:

\begin{tblr}{
 colspec={|X|X|X[2]|X|},
 row{1}={font=\bfseries},
 rowhead=1,
 hlines,
 }
 To Do This               & Motion           & You Say This                                                                 & Votes Required \\
 Change Wording of Motion & Amend            & ``I move to amend the motion by\ldots'' (Adding, Striking Out, Substituting) & Majority       \\
 Send Motion to Committee & Commit           & ``I move that the motion be referred to\ldots''                              & Majority       \\
 Postpone Motion          & Postpone         & ``I move that the motion be postponed to\ldots''                             & Majority       \\
 Limit Debate Time        & Limit Debate     & ``I move that debate be limited to\ldots''                                   & Two-thirds     \\
 End Debate               & Call to Question & ``I move to call the question.''                                             & Two-thirds     \\
\end{tblr}

\textbf{Amend}

Amend is the most frequently used and most important of the secondary
motions. There are three ways to amend a motion:

\begin{enumerate}
 \item
  To add words or phrases.
 \item
  To strike out words or phrases.
 \item
  To substitute by

  \begin{enumerate}
   \item
    striking out and inserting words;
   \item
    substituting an entire motion or paragraph.
  \end{enumerate}
\end{enumerate}

The first speakers to respond to a proposed amendment are the mover and
seconder of the main motion. If they choose to adopt the amendment, it
is deemed friendly and does not need to be voted upon and the main
motion is changed. The chair must then read the new main motion on the
floor. If the mover or seconder does not choose to adopt the amendment,
it is deemed unfriendly, and the chair must open debate and hold a vote
to adopt it.

\textbf{Commit}

To commit a motion sends the main motion on the floor to a committee so
that it can be carefully studied and put into proper form for the MES
Council to consider, and bring back the main motion to the MES Council
with a report of their findings.

\textbf{Postpone}

A motion to postpone delays action on a question until later in the same
meeting or until another specified meeting. This motion is useful when
information regarding the pending motion will be available at a later
time, a member realizes their delegation is not present for the vote, or
it is time for recess or adjournment.

\textbf{Limit Debate}

Limit debate is the motion by which the MES Council can exercise special
control over the debate by:

\begin{enumerate}
 \item
  Reducing the number and length of speeches allowed.
 \item
  Requiring that debate be limited to a period of time, at the end of
  which, the vote must be taken.
\end{enumerate}

\textbf{Call to Question}

Call to question is the motion used to cut off debate and to bring the
group to an immediate vote on the pending motion

\hypertarget{talking-system}{%
 \subsection{TALKING SYSTEM}
 \label{talking-system}}

\begin{enumerate}
 \item
  Should you have a new point to bring up in discussion, you should
  raise your hand as well as your index finger to be put on the speaking
  list.
 \item
  Should you have a direct response, you should raise your hand as well
  as your index and middle finger to be put on the speaking list. This
  takes precedence over new points, however it may only be used to make
  a direct response to something recently said. Such a direct response
  must stay on the same topic as the point to which it is responding, it
  is at the discretion of the chair to decide if the response does not
  meet this criteria.
 \item
  Should you have a point of clarification, you should raise your hand
  as well as your pinky finger. This takes precedence over all others on
  the speaking list, however it may only be used to seek clarification
  of something recently said. Such a point of clarification must be
  seeking a concise answer.
\end{enumerate}

\hypertarget{board-of-advisors-terms-of-reference}{%
 \section{\texorpdfstring{\emph{BOARD OF ADVISORS TERMS OF
     REFERENCE}}{BOARD OF ADVISORS TERMS OF REFERENCE}}
 \label{board-of-advisors-terms-of-reference}}

\begin{enumerate}
 \item
  Objective

  The Board of Advisors exists to provide strategic guidance and a
  thoughtful sounding board for the MES Executive and other MES members as
  appropriate.

  The Board of Advisors is a group of advisors, rather than a Board of
  Directors which aims to form consensus or decisions. Its powers will be
  limited to providing advice and recommending lines of action. The advice
  is available both as requested by the MES Executive and also by the
  Board of Advisors proactively contacting the MES Executive with ideas,
  leads, and opportunities.

 \item
  Membership

  Members shall be selected based on their position within a particular
  group (e.g. Faculty of Engineering)or based on their proven commitment
  and passion for the MES.

  Members may be asked to leave by the President if they fail to fulfill
  the responsibilities outlined within these terms of reference. If the
  Board loses a member due to job change or attrition, the President will
  prioritize filling this position with the appropriate candidate.

  The Board of Advisors shall be comprised of the following members:

  \begin{enumerate}
   \item
    Associate Dean of Engineering (Academic)
   \item
    Manager of Alumni Relations \& Youth Programs Office
   \item
    Faculty of Engineering Faculty Member
   \item
    Faculty of Engineering Staff Member
   \item
    Any former MES President
  \end{enumerate}

  The Board of Advisors shall have the following ex-officio members:

  \begin{enumerate}
   \item
    The previous MES President
  \end{enumerate}

  Members of the board shall:

  \begin{enumerate}
   \item
    Be selected to provide the mix of expertise necessary to best guide
    the MES Executive towards the MES' Mission Statement.
   \item
    Serve in their individual capacity, regardless of their relationship
    with a specific group (e.g. their employer) for a renewable term of
    one year.
   \item
    Participate in at least one meeting per year.
   \item
    Be appointed solely by the MES President every year.
  \end{enumerate}


 \item
  Meetings

  Board of Advisors meetings shall:

  \begin{enumerate}
   \item
    Occur at least twice per year, typically once in each semester. One
    meeting will serve as a transition meeting with both incoming and
    outgoing MES Executive members present. Additional meetings may also
    be held under special circumstances.
   \item
    Be located on McMaster University campus if possible.
   \item
    Have a quorum consisting of at least 50\% of the Faculty of
    Engineering representatives and 50\% of the alumni representatives.
   \item
    Be organized by the Manager of Alumni Relations \& Youth Programs
    Office.
   \item
    Have minutes taken by the MES VPF.
  \end{enumerate}


 \item
  Responsibilities

  The Board of Advisors will:

  \begin{enumerate}
   \item
    Meet with the MES Executive to review progress, address strategic
    questions, and plan for the upcoming year.
   \item
    Assist transitions between the incoming and outgoing MES Executives by
    maintaining knowledge throughout the years.
   \item
    Guide the MES Executive towards all of the MES' long-term goals.
   \item
    Represent the MES' best interests across campus and in industry.
   \item
    Act as stewards of the MES Mission Statement to help ensure that the
    MES Executive is appropriately advancing its mission.
   \item
    Engage in a reflective, self-evaluative process to improve board
    effectiveness.
   \item
    Act as a spokesperson on behalf of the MES, which could include being
    profiled on the MES website and in other communication as a Board of
    Advisors member.
   \item
    Declare any individual conflicts of interest to the MES President.
   \item
    Ensure the next meeting is planned at an appropriate time.
  \end{enumerate}
\end{enumerate}
% \end{document}