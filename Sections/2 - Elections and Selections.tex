\section{Elections \& Selections}
\label{elections-selections}
\subsection{General}
\label{elections-general}
\begin{enumerate}
 \item
  There shall be two classes of internal MES elections: regular elections and by-elections.
 \item
  All MES elections shall be administered and overseen by the Elections Committee (see MES Bylaws Section \ref{elections-committee}).
 \item
  Election dates will be determined at the discretion of the Elections Committee.
 \item
  At least 3\% of all eligible McMaster undergraduate students in the Faculty of Engineering must vote for an election to be considered valid.
 \item
  Elections shall be held online and officiated by the Faculty of Engineering.

\end{enumerate}

\subsection{Regular Elections}
\label{regular-elections}
\begin{enumerate}
 \item
  A meeting mandatory for all candidates will be held by the Chief Returning Officer prior to the campaign period of each regular election to inform candidates of the campaign rules and important dates as determined by the Elections Committee.

  \begin{enumerate}
   \item
    The consequences for an unexcused absence from this meeting on the part of a candidate shall be decided on a case-by-case basis by the Elections Committee but may include removal from the election if the Elections Committee sees fit.
  \end{enumerate}
 \item
  There shall be four distinct types of regular elections: presidential elections, general elections, Bachelor of Technology elections, and First Year Representative elections.
 \item
  A ``meet-and-greet'' style event will be held by the Chief Returning Officer during the campaign period of every regular election, except for First Year Representative elections.

\end{enumerate}

\subsubsection{Presidential Elections}
\label{presidential-elections}
\begin{enumerate}
 \item
  The presidential elections shall serve to elect the MES President.
 \item
  Presidential elections are to be held prior to Fireball, and the winning presidential candidate shall be announced at Fireball.
 \item
  All McMaster undergraduate students in the Faculty of Engineering shall be eligible to vote for the MES President.

\end{enumerate}

\subsubsection{Bachelor of Technology Election}
\label{bachelor-of-technology-election}
\begin{enumerate}
 \item
  The Bachelor of Technology elections shall serve to elect the Bachelor of Technology Representative.
 \item
  Bachelor of Technology elections are to be held in November, with polls closing prior to the last day of Fall term classes.
 \item
  Only the students currently enrolled in the Bachelor of Technology program shall be eligible to vote for the Bachelor of Technology Representative.
\end{enumerate}

\subsubsection{General Elections}
\label{general-elections}
\begin{enumerate}

 \item
  The following MES Council members shall be elected at the time of the a general election:

  \begin{enumerate}
   \item
    Vice President, Student Life
   \item
    Vice President, External Relations
   \item
    Vice President, Academic
   \item
    Vice President, Finance
   \item
    Vice President, Communications
   \item
    Vice President, Internal
   \item
    Associate Vice President, Events
   \item
    Associate Vice President, Clubs
   \item
    Associate Vice President, Academic Resources
   \item
    Chemical Engineering Representative
   \item
    Civil Engineering Representative
   \item
    Computer Science Representative
   \item
    Electrical and Computer Engineering Representative
   \item
    Engineering and Management Representative
   \item
    Engineering and Society Representative
   \item
    Engineering Physics Representative
   \item
    Integrated Biomedical Engineering and Health Sciences Representative
   \item
    Materials Science and Engineering Representative
   \item
    Mechanical Engineering Representative
   \item
    Mechatronics Engineering Representative
   \item
    Software Engineering Representative
  \end{enumerate}
 \item
  The general election shall be completed a week before the second term General Meeting at the latest.
 \item
  All McMaster undergraduate students in the Faculty of Engineering shall be eligible to vote for all the positions listed above except for the Program Representatives. Only the aforementioned students currently enrolled in the corresponding program shall be eligible to vote for their Program Representatives.

\end{enumerate}

\subsubsection{First Year Representative Elections}
\label{first-year-representative-elections}
\begin{enumerate}
 \item
  Six First Year Representatives shall be elected from the first year class of McMaster undergraduate students in the Faculty of Engineering.

  \begin{enumerate}
   \item
    Only first year students enrolled in the McMaster Faculty of Engineering are eligible to vote. These students shall only be eligible to vote for the representatives of the program that they are enrolled in
  \end{enumerate}
 \item
  There shall be a nomination period of one week during which all interested first year students may submit their nomination forms to the Chief Returning Officer.
 \item
  Elections shall be held in September.
\end{enumerate}
\subsection{By-Elections}
\label{by-elections}

\begin{enumerate}
 \item
  By-elections may be held to fill a vacancy in an elected office due to a reason other than the expiration of the term of office of the official in question.

  \begin{enumerate}
   \item
    A by-election need not be held to fill a vacancy if a position becomes vacant within the last eight weeks of the term of office of the elected official.
  \end{enumerate}
 \item
  By-elections will be conducted with their own set of rules as follows:

  \begin{enumerate}
   \item
    The nomination period shall last a minimum of 5 days (including weekend days)
   \item
    The campaign period shall last a minimum of 5 weekdays
   \item
    Polling shall immediately follow the campaign period
  \end{enumerate}
\end{enumerate}

\subsection{Eligibility for Elected Positions}
\label{eligibility-for-elected-positions}

\begin{enumerate}
 \item
  Candidates are not required to be MES members at the time of their campaign, but if elected they must become full members of the MES as detailed in the Constitution and must remain members for the duration of their term.

  \begin{enumerate}
   \item
    Individuals may hold an elected position on the MES without holding an MES membership throughout their term. This exception shall be granted on a financial need basis at the discretion of the Chief Returning Officer.
  \end{enumerate}
 \item
  Candidates for President must be registered in Level III or higher at the time of election.
 \item
  Candidates for VPSL, VPA, VPX, VPC, VPI, and VPF must be enrolled in Level II or higher at the time of election.
 \item
  Candidates for Program Representatives must be enrolled in the program they are campaigning for at the time of election

  \begin{enumerate}
   \item
    In the case a program representative changes their enrolment program or resigns from their position, the seat will be immediately vacated. The Elections Committee will determine the election process for the vacated position.
  \end{enumerate}
 \item
  In the case of a by-election for a Vice-President or Program Representative during the academic year, candidates must be enrolled in Level II or higher at the time of election.
 \item
  Candidates for First Year Representative must be registered in Level I at the time of the election or by-election for that position.
 \item
  All candidates must follow the rules set down in these bylaws, or in any other duly enacted document, and any rulings issued by the Elections Committee.
 \item
  Eligibility exceptions can be made at the discretion of the Elections Committee.
\end{enumerate}

\subsection{Eligibility for Appointed Positions}
\label{eligibility-for-appointed-positions}

\begin{enumerate}
 \item
  Chief Returning Officer candidates must have held at least one voting or appointed position on the MES Council for a complete term.
 \item
  At least one Culture Coordinator(s) must be registered in Level III or higher at the time of their appointment.
 \item
  Co-Orientation Coordinators must meet eligibility requirements, as outlined in \ref{co-orientation-coordinators-eligibility}).
\end{enumerate}

\subsection{Election Dates}
\label{election-dates}

\subsubsection{Election Dates}
\label{election-dates-1}
All Society elections shall consist of three distinct periods of time:
the nomination period, the campaign period and the election period.

\begin{enumerate}
 \item
  The nomination period for all MES elections shall extend over at least 10 full days, during which classes are in session. Duly completed nomination forms shall only be accepted by the Chief Returning Officer during this period.
 \item
  The campaign period for the presidential and general elections shall span a period of no less than 5 full weekdays during which classes are held. The weekend days enclosed by the campaign commencement and termination dates will be considered open for campaigning.
 \item
  It is recommended that at least one day be scheduled between the close of nomination period and the commencement of campaign period.
 \item
  The First Year Representative elections' campaign period shall span a minimum of 4 full weekdays during which classes are held. The weekend days enclosed by the campaign commencement and termination dates will be considered open for campaigning.
 \item
  All casting of ballots for all MES elections shall be conducted immediately following the campaign period on a day in which classes are held. The McMaster undergraduate students in the Faculty of Engineering of the appropriate constituencies shall have an opportunity to cast ballots over a period of no less than five consecutive hours on the election day.

  \begin{enumerate}
   \item
    For online polling, the balloting period must extend for a minimum of eighteen consecutive hours.
  \end{enumerate}
 \item
  Election dates are set at the discretion of the Elections Committee. % this item is redundant - see elections-general

\end{enumerate}
\subsection{Term of Office}
\label{term-of-office}
\begin{enumerate}
 \item
  The term of office for the Associate Vice-Presidents and all Program Representatives, with the exception of the First Year and Bachelor of Technology Representatives, shall commence the day of the General Meeting, to be held in March of the year of election, upon ratification and terminate the day of the General Meeting to be held in March of the following year, upon ratification of the incoming MES Council.
 \item
  The term of office of the First Year Representatives shall commence at the first MES Council meeting following their election, pending ratification by the MES Council, and terminate on the last day of class of that academic year.
 \item
  The term of office of the Bachelor of Technology Representative shall commence at the first MES Council meeting after the conclusion of their election, pending ratification by the MES Council, and shall terminate after the election of the next Bachelor of Technology Representative in the following year.
 \item
  The term of office for the Executive shall commence the day of the General Meeting, to be held in March of the year of election, upon ratification, and terminate on August 31 of the following year.
  \begin{enumerate}
   \item
    The incoming President shall be referred to as the President-Elect, from their election in January until their ratification at the General Meeting in March.
   \item
    The Executive shall hold voting power commencing the day of the General Meeting, to be held in March of the year of election, upon ratification, and terminating the day of the General Meeting to be held in March of the following year, upon ratification of the incoming Executive.
  \end{enumerate}

 \item
  An MES Council member cannot hold more than one voting position, on the MES Council, simultaneously. Should this occur, the MES Council member in question must resign from one of the positions.

\end{enumerate}
\subsection{Nominations}
\label{nominations}
\begin{enumerate}
 \item
  All eligible McMaster undergraduate students in the Faculty of Engineering (see MES Bylaws Section C.1.1) wishing to run for any MES Council position open for election shall present an appropriate nomination form (see Appendix G) signed by the relevant amount of McMaster undergraduate students in the Faculty of Engineering, of the constituency in question, to the Chief Returning Officer during the nomination period.
  \begin{enumerate}
   \item
    All executive and associate vice president positions (President, VPs, AVPs) are required to have a minimum of 20 nomination signatures.
   \item
    All elected program representative positions are required to have a minimum of 10 nomination signatures.
  \end{enumerate}
 \item
  If, at the conclusion of the nomination period there are no nominations for an MES Council position open for election, the Chief Returning Officer will make arrangements for by-elections to fill the vacancy.

  \begin{enumerate}
   \item
    If, at the conclusion of the nomination period there is only one nomination for any MES Council position, an election for that position must still be held.
  \end{enumerate}
 \item
  It is the responsibility of the Chief Returning Officer to ensure the publication of elected position nominees, in a timely fashion. This may be published to a page on the MES website, the MES Facebook page, or to another public forum accessible to any McMaster undergraduate students in the Faculty of Engineering.

\end{enumerate}

\subsection{Campaign}
\label{campaign}

\subsubsection{Campaign Conduct}
\label{campaign-conduct}
\begin{enumerate}
 \item
  All candidates must conduct their campaign according to the campaign rules as determined and set out by the Elections Committee.
 \item
  Any violation of these rules will be reviewed by the Elections Committee and subject to its ruling, which may include disqualification.
 \item
  Campaigning shall be defined as, but not limited to:

  \begin{enumerate}
   \item
    Distribution of campaign materials;
   \item
    Speaking to classes, residences, student groups, or individuals for the purpose of presenting a platform as a candidate for a position;
   \item
    Sharing of website links or social media pages that are relevant to campaigns;
   \item
    Any action performed primarily for the purpose of seeking election.
  \end{enumerate}
 \item
  Campaign material is defined as, but is not limited to:

  \begin{enumerate}
   \item
    Campaign posters
   \item
    Campaign swag/apparel
   \item
    Social media pages or events, including any relevant posts, photos or documents contained therein
   \item
    Personal campaign websites
  \end{enumerate}
 \item
  To promote the election and increase voter turnout, all campaign material must include:

  \begin{enumerate}
   \item
    the MES logo
   \item
    the date and location of polling
   \item
    the position for which the candidate is seeking election
  \end{enumerate}
 \item
  All campaigning shall be in good taste (no promoting violence, substance abuse or discrimination) and fairly conducted with courtesy to other candidates. No slander or libel will be tolerated.

\end{enumerate}

\subsubsection{Closure of Campaign Period}
\label{closure-of-campaign-period}
\begin{enumerate}
 \item
  Candidates are to cease campaigning once the campaign period has ended and voting has begun.
 \item
  Once the campaign period has ended, no more campaign material may be posted by the candidate or their campaign team. Any campaign materials posted prior to the closing of the campaign period may remain visible and available in order to keep voters informed of candidates and their intentions. The candidates are still expected to abide by the campaign rules, determined by the Elections Committee.
 \item
  Campaign posters must be removed within 24 hours after the polls close.
\end{enumerate}

\subsubsection{Campaign Team}
\label{campaign-team}
\begin{enumerate}
 \item
  Candidates choosing to recruit a campaign team must submit names and emails of all members of their campaign team to the Elections Committee.
 \item
  Campaign team members are the responsibility of the candidate, and must follow all election rules and procedures.
 \item
  At the discretion of the Elections Committee, MES members that are actively campaigning for a specific candidate will be added to that candidate's team.

\end{enumerate}

\subsubsection{Campaign Posters}
\label{campaign-posters}
\begin{enumerate}
 \item
  The content and placement of the campaign posters should follow MSU guidelines.
 \item
  The Chief Returning Officer must vet and approve all campaign posters to ensure the required information is included.
 \item
  The Chief Returning Officer shall be responsible for obtaining the MSU Underground's approval of all campaign posters.
 \item
  Only masking tape, regular staples and tacks are used to put up posters on appropriate boards. Only masking tape shall be used to put up posters on appropriate walls. Duct tape, packing tape, glue, staple guns, etc. are prohibited.
 \item
  Candidates may not use physical posters or banners that exceed 11'' x 17'' inches.
 \item
  Upon submission of receipts, up to a maximum of \$5 will be reimbursed to candidates. This is to ensure that all candidates have equal opportunity to campaign via printed posters.
\end{enumerate}

\subsubsection{Online and Social Media Campaigning}
\label{online-and-social-media-campaigning}
\begin{enumerate}
 \item
  All digital campaign materials must follow the guidelines of campaign conduct.
 \item
  The Elections Committee may modify the rules of social media campaigning, as long as notice is duly given to candidates and the bylaws are amended accordingly at subsequent Council meetings.
 \item
  Digital campaigning must be carried out in a manner where candidates do not spam or otherwise make their candidacy announced without direct interaction from potential voters. Digital campaigning must also minimize the leverage of a candidate's existing connections and networks. This includes, but is not limited to, the following:

  \begin{enumerate}
   \item
    Creating new, alternate, or throwaway accounts
   \item
    Modifying the handle, tag, username, status, or equivalent to refer to their candidacy
   \item
    Receiving endorsements or having posts reshared by MES Groups
  \end{enumerate}
 \item
  Digital campaigning must be carried out in a manner where candidates do not campaign in group chats or any situation where it would be considered spam if all candidates proceeded to campaign in a similar manner. This would include, but is not limited to:

  \begin{enumerate}
   \item
    Announcing candidacy on Reddit, Discord, or other platforms
   \item
    Announcing candidacy in academic or social group chats
  \end{enumerate}
 \item
  Approved digital campaigning includes, but is not limited to the following:

  \begin{enumerate}
   \item
    Creating a personal webpage or website
   \item
    Changing the viewability of your account to public
   \item
    Modifying the bio, about me, or equivalent section of profile, given that potential voters cannot view this without directly navigating to your profile.
   \item
    Sharing posts created by the MES
   \item
    Creating posts, memes, stories, reels, videos, or other content
   \item
    Asking Discord server moderators to announce MES Elections are happening, given there is no acknowledgement of the candidate
  \end{enumerate}
 \item
  The Elections Committee can approve any digital campaign materials by request of a candidate. Candidates are encouraged to err on the side of caution.

\end{enumerate}

\subsubsection{Other Campaign Procedures}
\label{other-campaign-procedures}
\begin{enumerate}
 \item
  Presidential candidates may submit a statement no longer than 500 words to the Chief Returning Officer by the end of the first day of the campaign period. These statements will be posted on the MES website and on other MES social media.
 \item
  Candidates in general elections may submit a statement no longer than 250 words to the Chief Returning Officer by the end of the first day of the campaign period. These statements will be posted on the MES website and on other MES social media.
 \item
  Candidates must submit all expenses made during the election to the Chief Returning Officer, with receipts. The total amount of campaign expenses per candidate may not exceed \$50 during any election. Candidates shall not be reimbursed for any campaign expenses, except for the \$5 poster printing stipend, as outlined above in MES Bylaws Section \ref{campaign-posters}. % removed closing bracket
 \item
  When candidates are campaigning in lectures they shall consult the professor prior to the class and must get permission to do the campaign presentation. % comma

\end{enumerate}

\subsection{Elections Committee}
\label{elections-committee-guidelines}
\begin{enumerate}
 \item
  The Elections Committee shall consist of the Chief Returning Officer as a non-voting chair, the Equity and Inclusion Officer, the VPSL, and the President. If any member of the Elections Committee is seeking office in an MES election or demonstrates an obvious bias to any one candidate, the MES Executive shall appoint an unbiased MES Council member who is not seeking election. % removed extra space and comma
 \item
  The Elections Committee shall meet prior to the first of September of each year to set the dates for all elections. These dates shall be made ready for advertisement to all McMaster undergraduate students in the Faculty of Engineering once the dates are set.
 \item
  For all MES elections, the Elections Committee shall:

  \begin{enumerate}
   \item
    Post a list of all available positions to be filled and a summary of their duties prior to the nomination period.
   \item
    Post all election notices on the MES website, and on other social and physical media.
   \item
    Review and update the campaign rules and inform candidates of the campaign rules during their respective nomination periods.
   \item
    Pass judgment as to the eligibility of all the candidates (see MES Bylaws Section \ref{eligibility-for-elected-positions}), and ensure that correct nomination procedures have been followed.
   \item
    Ensure that all campaign materials are in accordance with MSU regulations.
   \item
    Not engage in the campaign of any candidate(s).
   \item
    Be responsible for drawing up the official ballot for all MES elections.
   \item
    Be responsible for running all elections in accordance with the MES Bylaws Section \ref{elections-selections}.
   \item
    Post the names of all successful candidates as soon as they are known. Except for presidential candidates, which will be announced at Fireball.
   \item
    Take any corrective or disciplinary actions necessary with regard to electoral rule violations of any type, as outlined in this or any other duly enacted document, with the provision that the MES Executive, excepting any members thereof directly affected by the matter in question, shall act as the final arbitrator in all disputes.
   \item
    Prepare a post-election report detailing the precise results of the election and the nature of actions taken in regard to any protests, disputes, and/or rule contraventions during the course of the election. Such a report must be prepared after every MES election and submitted to the MES Council.
  \end{enumerate}
\end{enumerate}

\subsection{Election Procedures}
\label{election-procedures}
\begin{enumerate}
 \item
  Voting for elections may be conducted online. If elections are held online and more than one candidate is running for the same position, a single transferable voting system shall be used.
 \item
  An election shall be declared invalid if:

  \begin{enumerate}
   \item
    The Elections Committee rules that the number and seriousness of election rule and procedure infractions that may have affected the election results warrants the invalidation of the election.
  \end{enumerate}
 \item
  There must be both a non-confidence and an abstain voting option on the ballots of all MES elections.
 \item
  The candidate receiving the greatest number of valid votes for a given position shall be declared elected to that position, subject to ratification.
 \item
  In the case of a tie, the Chief Returning Officer shall reopen the campaign period for the tied position for a period of at least four days. Only those candidates who tied will be allowed to campaign again. An additional voting day, immediately following the second campaign period, will be opened up to all McMaster undergraduate students in the Faculty of Engineering of the constituency in question
 \item
  In the event that a member of the MES will not be able to vote on the day of an election, they may vote in advance. It is the responsibility of the voter to make proper arrangements as listed below:

  \begin{enumerate}
   \item
    An advance vote must be made in the presence of the Chief Returning Officer and one other Elections Committee member.
   \item
    After the vote is completed, the Chief Returning Officer and Elections Committee member will both initial the ballot and place it in the advance poll envelope.
   \item
    Any ballot placed in the advance poll envelope shall not be interfered with in any way. The ballots must only be removed from the envelope while the entire Elections Committee is present, to add the votes as recorded on the ballots to the election results.
   \item
    The advance poll envelope shall be the responsibility of the Chief Returning Officer
  \end{enumerate}
\end{enumerate}