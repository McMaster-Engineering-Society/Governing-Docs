\section{Groups: Clubs, Teams and Program Societies}
\label{groups-clubs-teams-and-program-societies}

\subsection{Program Societies}
\label{program-societies}
\begin{enumerate}
 \item
  A program society shall be a fully recognized organization of McMaster undergraduate students, in the Faculty of Engineering. Societies shall be affiliated with either a specific engineering program.
 \item
  The MES recognizes the following program societies:

  \begin{enumerate}
   \item
    Bachelor of Technology Association (BTA)
   \item
    Chemical Engineering Society (ChES)
   \item
    Civil Engineering Society (CSS)
   \item
    Computer Science Society (CSS)
   \item
    Electrical and Computer Engineering Society (ECES)
   \item
    Engineering and Society Students' Association (ESSA)
   \item
    Engineering Physics Society (EPS)
   \item
    iBioMed Society (iBMS)
   \item
    Materials Science and Engineering Society (MSES)
   \item
    McMaster Engineering and Management Society (MEMS)
   \item
    McMaster Mechatronics Society (MMS)
   \item
    McMaster Society of Mechanical Engineering (MSME)
   \item
    Software Engineering Society (SES)
  \end{enumerate}
 \item
  All societies shall report their financial statement and receive funding as outlined in the Clubs Financial Policy (see MES Bylaws Section \ref{clubs-and-teams-funding-policy})
 \item
  All society presidents will be members of the Club Leaders Committee (see MES Bylaws Section \ref{club-leaders-committee}).
 \item
  New MES Program Societies must put forth a motion for recognition by the MES at a General Meeting.

\end{enumerate}

\subsection{Clubs}
\label{clubs}
\begin{enumerate}
 \item
  A Club shall be an organization composed of at least 75\% McMaster undergraduate students in the Faculty of Engineering.
 \item
  A Club may receive funding from the MES by following the procedures set out in MES Bylaws Section \ref{clubs-and-teams-funding-policy}.
 \item
  The Clubs that are officially recognized by the MES are:

  \begin{enumerate}
   \item
    Competitive Programming, McMaster (MCP)
   \item
    Deep Space Analogue Research Expedition, McMaster (DARE)
   \item
    DeltaHacks
   \item
    Design League, McMaster (MDL)
   \item
    Engineering Musical, McMaster (MEM)
   \item
    Engineers Without Borders (EWB)
   \item
    EngiQueers, McMaster (EQ)
   \item
    Google Developer Student Clubs, McMaster (GDSC)
   \item
    HealthHatch
   \item
    Heavy Construction Student Chapter, McMaster (HCSC)
   \item
    Institute of Electrical and Electronics Engineers, McMaster Student Branch (IEEE)
   \item
    Institute of Transportation Engineers Student Chapter, McMaster (ITE)
   \item
    McMaster Energy Association (MEA)
   \item
    McMaster Engineering Hockey Club
   \item
    McMaster Engineers with Disabilities
   \item
    McMaster Society for Engineering Research (MacSER)
   \item
    Mechanical Contractors Association Hamilton Niagara, McMaster Student Chapter (MCAHN)
   \item
    Medical Engineering Design Team, McMaster (MED-T)
   \item
    National Society of Black Engineers, McMaster Chapter (NSBE)
   \item
    North American Young Generation in Nuclear, McMaster (NAYGN)
   \item
    Pumpkin Chunkin Club, McMaster Engineering (MPC)
   \item
    Start Coding, McMaster
   \item 
    Sumobots, McMaster
   \item
    Women in Engineering, McMaster (WiE)
  \end{enumerate}
\end{enumerate}

\subsection{Teams}
\label{teams}
\begin{enumerate}
 \item
  A Group is considered to be a Team if they are a non-sporting organization composed of at least 75\% McMaster undergraduate students in the Faculty of Engineering which competes at external events as representatives of McMaster.
 \item
  In order to receive Team status, teams must have competed in at least one external event the previous year, or be approved by the MES council for extenuating circumstances. Otherwise, the group will be transferred to Club status.
 \item
  Teams that are officially recognized by the MES are:

  \begin{enumerate}
   \item
    Aerospace Team, McMaster (MAST)
   \item
    Baja Racing, McMaster
   \item
    Chem E Car, McMaster
   \item
    Concrete Toboggan, McMaster (MECTT)
   \item
    EcoCar, McMaster
   \item
    Formula Electric, McMaster (MACFE)
   \item
    Mars Rover Team, McMaster (MMRT)
   \item
    RoboMaster Team, McMaster
   \item
    Rocketry, McMaster
   \item
    Seismic Design Team, McMaster (MSDT)
   \item
    Solar Car, McMaster
   \item
    Steel Bridge Team, McMaster (MSBT)
  \end{enumerate}
 \item
  Team projects should involve the application of engineering design concepts.

\end{enumerate}

\subsection{Affiliates}
\label{affiliates}
\begin{enumerate}
 \item
  An affiliate shall be an organization composed of at least 75\% McMaster undergraduate students in the Faculty of Engineering.
 \item
  An affiliate is a collection of people who wish to be recognized by the MES and do not require funding.
 \item
  Affiliates are a stepping stone for new groups to eventually reach ratification as a club or team.
 \item
  Affiliates that are officially recognized by the MES are:

  \begin{enumerate}
   \item
    American Indian Science and Engineering Society, McMaster Chapter (AISES)
   \item 
    BioDesign, McMaster
   \item
    Brewing Club, McMaster
   \item
    Concrete Canoe, McMaster
   \item
    Engineering Jazz Band, McMaster
   \item
    Mac Quantum Club
   \item 
    McMaster Advanced Space Systems (MASS)
   \item
    McMaster Interdisciplinary Satellite Team (MIST)
  \end{enumerate}
\end{enumerate}

\subsection{Ratification Process}
\label{ratification-process}
\begin{enumerate}
 \item
  To become an MES Affiliate, you must email the DoC your intent to become an MES Affiliate.
 \item
  After meeting with the DoC, all intending MES Affiliates must put forth a motion for affiliation at a Council Meeting. Council will vote on the motion and the DoC will assist you with the next steps whatever the result!
 \item
  To become an MES Group, you must be an MES Affiliate for at least one year, then let the DoC know your intent to apply to be an MES Group.
 \item
  Each MES Affiliate intending to become an MES Group must present a motion to MES Council to be recommended to the General Meeting. Upon a majority vote from Council, the MES Affiliate must submit a motion for ratification at the second General Meeting in a given academic year (See MES Bylaws Section \ref{general-meetings-sagm}).

  \begin{enumerate}
   \item
    On a majority vote at the General Meeting, the MES Affiliate will be ratified as a Group recognized by the MES.
   \item
    Motions may be put forth at the first General Meeting in a given academic year only by direct permission from the DoC.
  \end{enumerate}
 \item
  Long-established groups may be able to bypass the affiliation stage and become directly ratified as an MES Group.

  \begin{enumerate}
   \item
    This follows the same ratification process for MES Affiliates, wherein Council will motion whether to send the group to the Semi Annual General Meeting for ratification.
  \end{enumerate}

\end{enumerate}

\subsection{De-Ratification Process}
\label{de-ratification-process}
If a group is not performing or achieving certain standards, they may be
de-ratified.

\begin{enumerate}
 \item
  The de-ratification process will occur at the Winter General Meeting.
 \item
  The DoC shall be responsible for compiling a list of groups that shall undergo the de-ratification process. A motion for de-ratification may be proposed at the discretion of the DoC, but reasons could include that the group:

  \begin{enumerate}
   \item
    Does not present at either General Meeting in that year, nor explains their absence to council
   \item
    Does not attend ClubsFest, or any outreach events during the academic year
   \item
    Does not host events or maintain any social media presence during the academic year
  \end{enumerate}
 \item
  The DoC will be responsible for communicating with groups that may be facing de-ratification.
 \item
  The DoC will present to Council all groups that shall under-go the de-ratification process, and, upon approval from Council, these groups shall be presented and voted for de-ratification at the Winter General Meeting.
 \item
  Clubs, Teams, and Affiliates may all be subject to the de-ratification process
 \item
  Recognized groups will be deratified if they are disbanded by the organization they are ratified by.
\end{enumerate}
% \end{document}
