\section{MES Robert's Rules of Order}
\label{mes-roberts-rules-of-order}
The following are meeting procedures to be followed in General Meetings
and MES Council meetings. Robert's Rules of Order is a strict, but
practical system for running meetings. The MES follows its own version
of Robert's Rules of Order as outlined below.

\subsection{ELEVEN FUNDAMENTAL RULES OF PROCEDURE}
\label{eleven-fundamental-rules-of-procedure}

\begin{enumerate}
 \item
  Principles

  \begin{enumerate}
   \item
    The right of majority to decide
   \item
    The right of minority to be heard
   \item
    The right of individual members
   \item
    The right of absentees
  \end{enumerate}
 \item
  All voting members of the MES Council are equal and their rights are
  equal. These rights are:

  \begin{enumerate}
   \item
    To attend council meetings
   \item
    To speak
   \item
    To move motions
   \item
    To second motions
   \item
    To vote
   \item
    To concede their rights to others
  \end{enumerate}
 \item
  All non-voting members of the MES Council are equal and their rights
  are equal. These rights are:

  \begin{enumerate}
   \item
    To attend council meetings
   \item
    To speak
   \item
    To movemotions % clerical
   \item
    To second motions
   \item
    To concede their rights to others
  \end{enumerate}
 \item
  All McMaster undergraduate students in the Faculty of Engineering are
  equal and their rights are equal. These rights are:

  \begin{enumerate}
   \item
    To attend council meetings
   \item
    To speak
   \item
    To move motions
  \end{enumerate}
 \item
  The chair has the following powers:

  \begin{enumerate}
   \item
    To speak when clarification is necessary
   \item
    To determine the speaking order
   \item
    To count votes
   \item
    To interpret the MES Policy Manual, Bylaws, and Constitution
   \item
    To recognize members
   \item
    To decide what is in order
   \item
    To remove members from council dependent on a two thirds majority
    vote of council. % clerical missing hyphenation
  \end{enumerate}
 \item
  The rights of MES supersede the rights of individual members: Should a
  conflict arise between the rights of the MES and the rights of an
  individual member, the rights of MES shall take precedence.
 \item
  Quorum must be present at MES Council meetings for business to be
  done. Quorum shall consist of half of all voting members of the MES
  Council as outlined by the MES Constitution, unless otherwise stated
  in the MES Policy Manual, Bylaws, or Constitution.
 \item
  It is the responsibility of every voting member of the MES Council to
  vote. If they do not vote, it shall be assumed that they are
  abstaining from the vote.
 \item
  One speaker at a time: Only one speaker recognized by the chair has
  the right to talk at any given moment.
 \item
  Personal remarks are always out of order, the chair maintains the
  right, by a two thirds majority, to remove members from meetings for
  failing to comply with this rule.
 \item
  If there is ever a dispute on a decision/interpretation the chair has
  made, any voting member of the MES Council may move to challenge the
  chair and their decision. This motion, which would require a seconder,
  would open up discussion to the floor about the
  decision/interpretation. Each member will be allowed to speak once to
  the decision/interpretation, closing with the chair defending their
  decision/interpretation. Following the chair's statement, there shall
  be a vote. If there is a two thirds majority in favor of overruling
  the chair's decision/interpretation, the chair will accept the
  decision as their own and move onwards or the chair will be required
  to relinquish the chair.

\end{enumerate}

\subsection{MOTIONS}
\label{motions}
There are two kinds of motions: Main Motions and Secondary Motions.

\subsubsection{Main Motions}
\label{main-motions}

A main motion is defined as a proposal that certain action be taken or
an opinion be expressed by the organization.

\begin{enumerate}
 \item
  All main motions must be submitted to the Administrator at least 24
  hours before the meeting at which it will be moved.
 \item
  A main motion brought toward the MES will be read to the MES Council
  by the chair.
 \item
  A main motion must be seconded in order to proceed to presentation
  and/or debate.
 \item
  Movers and seconders may each make a short presentation outlining the
  action.
 \item
  Debate is struck. The chair recognizes members who wish to speak by
  stating their names. In the case of multiple speakers, a cue will be
  established and decided by the chair.
 \item
  Debate should continue as long as members wish to discuss the question
  unless the chair has put the question to a vote or secondary motions
  have been adopted to either limit or close debate.
 \item
  When a main motion is put to question, the chair shall restate the
  motion. Voting members are instructed to raise their hand when those
  in favour, opposition, or abstention are called by the chair. Any
  singular vote may be noted by name, exclusively by request from the
  member to whom that vote belongs.
 \item
  Voting results are counted by the chair and Administrator, and the
  result of the main motion is announced to the MES Council by the
  chair.
 \item
  A main motion shall pass if the votes in favour are counted to be half
  plus one of the total votes, unless specifically stated otherwise in
  the MES Constitution, Bylaws, or Policy Manual.
 \item
  All motions must be in compliance with the MES Constitution, Bylaws,
  and Policy Manual as interpreted by the chair.

\end{enumerate}
\subsubsection{Secondary Motions}
\label{secondary-motions}
Secondary motions are motions that may be made while the main motion is
on the floor and before it has been decided. More than one motion can be
on the floor but only one main motion. All pending motions must relate
to the main motion on the floor, no new business may be introduced.

Secondary motions have rank among each other. They are arranged in a
specific order in which they must be considered and acted upon, some
motions taking precedence over others. The purpose is to avoid confusion
when they are applied to a main motion. A motion of higher rank can be
made at the time that a motion of lower rank is on the floor.

\subsubsection{THE RANKS OF SECONDARY MOTIONS}
\label{the-ranks-of-secondary-motions}

\begin{table}[h]
 \centering
 \begin{tblr}{cells={c},hlines={1pt, black}}
  CALL TO QUESTION                              \\
  LIMIT OR EXTEND LIMITS TO DEBATE OR QUESTIONS \\
  POSTPONE TO A SPECIFIC TIME                   \\
  COMMIT OR REFER                               \\
  AMEND                                         \\
  \textbf{MAIN MOTION}                          \\
 \end{tblr}
\end{table}

Rank can be symbolized by the rungs of a ladder. Those motions on the
lower rungs must yield to the motions on the rung or rungs above. For
instance call to question, which calls for an immediate vote, takes
precedence over all motions below it. All secondary motions listed in
the table above must be seconded and are debatable.

\subsubsection{WHAT DO I SAY?}
\label{what-do-i-say}

Secondary Motions Arranged From Lowest to Highest Rank:

\begin{tblr}{
 colspec={|X|X|X[2]|X|},
 row{1}={font=\bfseries},
 rowhead=1,
 hlines,
 }
 To Do This               & Motion           & You Say This                                                                 & Votes Required \\
 Change Wording of Motion & Amend            & ``I move to amend the motion by\ldots'' (Adding, Striking Out, Substituting) & Majority       \\
 Send Motion to Committee & Commit           & ``I move that the motion be referred to\ldots''                              & Majority       \\
 Postpone Motion          & Postpone         & ``I move that the motion be postponed to\ldots''                             & Majority       \\
 Limit Debate Time        & Limit Debate     & ``I move that debate be limited to\ldots''                                   & Two-thirds     \\
 End Debate               & Call to Question & ``I move to call the question.''                                             & Two-thirds     \\
\end{tblr}
\newline

\textbf{Amend}

Amend is the most frequently used and most important of the secondary
motions. There are three ways to amend a motion:

\begin{enumerate}
 \item
  To add words or phrases.
 \item
  To strike out words or phrases.
 \item
  To substitute by

  \begin{enumerate}
   \item
    striking out and inserting words;
   \item
    substituting an entire motion or paragraph.
  \end{enumerate}
\end{enumerate}

The first speakers to respond to a proposed amendment are the mover and
seconder of the main motion. If they choose to adopt the amendment, it
is deemed friendly and does not need to be voted upon and the main
motion is changed. The chair must then read the new main motion on the
floor. If the mover or seconder does not choose to adopt the amendment,
it is deemed unfriendly, and the chair must open debate and hold a vote
to adopt it.

\textbf{Commit}

To commit a motion sends the main motion on the floor to a committee so
that it can be carefully studied and put into proper form for the MES
Council to consider, and bring back the main motion to the MES Council
with a report of their findings.

\textbf{Postpone}

A motion to postpone delays action on a question until later in the same
meeting or until another specified meeting. This motion is useful when
information regarding the pending motion will be available at a later
time, a member realizes their delegation is not present for the vote, or
it is time for recess or adjournment.

\textbf{Limit Debate}

Limit debate is the motion by which the MES Council can exercise special
control over the debate by:

\begin{enumerate}
 \item
  Reducing the number and length of speeches allowed.
 \item
  Requiring that debate be limited to a period of time, at the end of
  which, the vote must be taken.
\end{enumerate}

\textbf{Call to Question}

Call to question is the motion used to cut off debate and to bring the
group to an immediate vote on the pending motion

\subsection{TALKING SYSTEM}
\label{talking-system}

\begin{enumerate}
 \item
  Should you have a new point to bring up in discussion, you should
  raise your hand as well as your index finger to be put on the speaking
  list.
 \item
  Should you have a direct response, you should raise your hand as well
  as your index and middle finger to be put on the speaking list. This
  takes precedence over new points, however it may only be used to make
  a direct response to something recently said. Such a direct response
  must stay on the same topic as the point to which it is responding, it
  is at the discretion of the chair to decide if the response does not
  meet this criteria.
 \item
  Should you have a point of clarification, you should raise your hand
  as well as your pinky finger. This takes precedence over all others on
  the speaking list, however it may only be used to seek clarification
  of something recently said. Such a point of clarification must be
  seeking a concise answer.
\end{enumerate}

