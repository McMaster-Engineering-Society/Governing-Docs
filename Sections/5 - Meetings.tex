 \section{\texorpdfstring{\emph{MEETINGS}}{MEETINGS}}
\label{meetings}

\subsection{General Meetings (SAGM)}
\label{general-meetings-sagm}
\begin{enumerate}
 \item
  There shall be at least one General Meeting per term.
 \item
  Quorum at General Meetings shall be 3\% of the total McMaster
  undergraduate students in the Faculty of Engineering.
 \item
  Quorum for each MES Program Society is ten persons per club. No one
  person can count towards the quorum of more than one MES Program
  Society.
 \item
  The Chief Returning Officer will chair the General Meetings. In the
  absence or declination of the Chief Returning Officer to chair, a vote
  shall be held to elect a chair with a two thirds majority vote.
 \item
  There shall be no proxy voting at General Meetings.
 \item
  Motions for the General Meeting are due one week prior to the meeting
  to allow for adequate advertising and review. It is at the discretion
  of the Chief Returning Officer to accept motions submitted after the
  deadline.
 \item
  The General Meeting agenda is to be posted at least 24 hours in
  advance of meeting.
 \item
  General Meetings shall be conducted in accordance with the MES version
  of Robert's Rules of Order (see ``MES Robert's Rules of Order'',
  Appendix I).

  \begin{enumerate}
   \item
    All McMaster undergraduate students in the Faculty of Engineering
    have the right to vote at MES General Meetings
  \end{enumerate}
\end{enumerate}

\subsection{MES Council Meetings}
\label{mes-council-meetings}

\subsubsection{General}
\label{meetings-general}

\begin{enumerate}
 \item
  MES Council meetings shall be held once every two weeks for the
  entirety of both semesters.
 \item
  Quorum will consist of two-thirds of all MES Council voting positions
  as is outlined within the MES Constitution. Unfilled First Year
  Representative positions shall not count towards quorum until the
  First Year Representative Election has taken place.
 \item
  The Administrator must be notified of any motions and agenda items at
  least 48 hours before the meeting. Motions and agenda items not
  submitted within this time period may not be permitted at the meeting,
  subject to decision by the MES Council.
 \item
  The Administrator must forward all motions for funding to the VPF upon
  receiving the motion(s) in order to provide time for review.
 \item
  The Administrator is responsible for recording the minutes of all
  proceedings of the meeting. If the Administrator is unavailable for
  all or part of a meeting, another chosen/volunteered MES Council
  member will record minutes in the Administrator's absence.
 \item \label{CRO-chair-meetings}
  The Chief Returning Officer shall chair all MES Council meetings. The
  President will chair any MES Council meetings in their absence. Should
  both the Chief Returning Officer and President decline or are unable
  to chair, a vote will be held to elect a new chair requiring a two
  thirds majority.
 \item
  MES Council meetings shall be conducted in accordance with the MES
  version of Robert's Rules of Order (see MES Bylaws Section \ref{mes-roberts-rules-of-order}.
 \item
  The adoption of the agenda will be motioned at the start of council
  meetings. Any amendments will be brought forth as motions to amend the
  agenda during this motion.
 \item
  The Administrator tallies and records any voting on motions. A second
  counter must confirm the Administrator's count.
 \item
  At the end of each MES Council meeting, before the motion to adjourn,
  there shall be an opportunity for new business to be added to the
  agenda pending a two thirds majority vote in favor to add the new
  business. New business must be in compliance with the restrictions of
  the Policy Manual, theConstitution, and this document.
 \item
  All submitted motions must follow the format outlined in Appendix Z
 \item
  MES Council is permitted to use a consent agenda, where multiple items
  of discussion can be grouped into a single motion and vote.
 \item
  MES Council may pass motions by unanimous consent, also known as
  general consent, when no elected Council member objects to a motion.
 \item
  The Chair is not required to pass the chair if they need to present,
  but must do so in the event where they are unable to moderate
  discussion in a proper and unbiased manner.
  \begin{enumerate}
    \item
     The Temporary Chair must:
     \begin{enumerate}
      \item
       be an elected Council member.
      \item
       be elected by majority vote.
      \item
       not pass the chair except when returning it to the Chair.
     \end{enumerate}
    \item
    The Chair automatically reassumes their role upon the conclusion of
    their presentation and discussion.
  \end{enumerate}

\end{enumerate}

\subsubsection{Attendance}
\label{attendance}
\begin{enumerate}
 \item
  Attendance at MES Council meetings is mandatory for all MES Council
  elected positions.
 \item
  Any absences must be communicated to the Chief Returning Officerand
  Administrator at least 24 hours in advance.
 \item
  Suitable excuses for missing meetings include:

  \begin{enumerate}
   \item
    Illness
   \item
    Classes
   \item
    Writing tests
   \item
    Representing the MES in an official capacity at another event
   \item
    Other emergencies (at the discretion of the Chief Returning Officer)
  \end{enumerate}
 \item
  If a council member is unable to attend they should appoint someone to
  be their proxy and seek the approval of the Chief Returning Officer
  (See MES Bylaws Section \ref{proxy-voting})
 \item
  If an MES Council member misses more than two MES Council meetings in
  a single semester without an approved excuse, they shall be removed
  from the MES Council at the discretion of the Chief Returning Officer.
  The position shall then be open for election or appointment
  accordingly.
\end{enumerate}

\subsubsection{Proxy Voting}
\label{proxy-voting}
\begin{enumerate}
 \item
  A voting Council member may proxy their vote at an MES Council meeting
  to another voting Council member or any Full Member of the MES.
 \item
  Approval of the proxy must be received from the Chief Returning
  Officer, and notice sent to the Administrator no less than 24 hours
  before the start of the meeting in question. A request to proxy will
  not be accepted at the start of or during any meeting.
 \item
  The proxy will remain in force for the duration of the selected
  meeting only.
 \item
  Notice of the proxy must be included on the meeting agenda and
  announced at the beginning of the meeting.
 \item
  Although the person initiating the proxy can give their voting
  preference to the voter, the person receiving the proxy can vote as
  they please. MES Council members should take this into consideration
  when choosing a person to vote in their absence.
\end{enumerate}

\subsubsection{Quorum}
\label{quorum}

\begin{enumerate}
 \item
  Quorum will consist of both:

  \begin{enumerate}
   \item
    60\% of voting Council members in attendance
   \item
    75\% of Full Quorum votes are present and binding

    \begin{enumerate}
     \item
      Full Quorum is the total number of elected council members. Full
      Quorum is normally 29 votes, unless voting positions are vacant.
     \item
      A vote is considered binding if it is held by the elected position
      or a binding proxy.
    \end{enumerate}
  \end{enumerate}
 \item
  Under no circumstance will a proxy be considered binding if the proxy
  is for the President, Vice-President, or an Associate-Vice-President.
 \item
  There are five classes of reasons for missing a Council meeting:

  \begin{enumerate}
   \item
    MES or Faculty representation absence (e.g. ESSCO/CFES Conference,
    MES Team external competition)
   \item
    Non-repeating, academic (e.g. midterms, industry night)
   \item
    Non-repeating, non-academic (e.g. concert, club event, illness)
   \item
    Repeating, academic (e.g. night classes)
   \item
    Repeating, non-academic (e.g. club meetings)
  \end{enumerate}
 \item
  In cases i) and ii), any proxy will be counted as binding.
 \item
  In cases iii) and iv), program representative proxies will be counted
  as binding only if the proxy is a member of their respective
  constituency.
 \item
  In case v), program representative proxies will be counted as binding
  only if the proxy is a member of their respective program society
  executive.
 \item
  In cases iii), iv) and v), first year representative proxies will be
  counted as binding only if the proxy is a member of their respective
  constituency.
 \item
  Unfilled positions shall not count towards quorum until the respective
  elections have taken place and the positions are filled.

\end{enumerate}

\subsection{Committee Meetings}
\label{committee-meetings}
\begin{enumerate}
 \item
  Committee Chair(s) shall be responsible for organizing and conducting
  regular meetings in an efficient and orderly manner (see MES Bylaws
  Section \ref{committees-general}).

\end{enumerate}

\subsection{Supervisory Meetings}
\label{supervisory-meetings}
\begin{enumerate}
 \item
  All MES Council members will meet with their respective supervisors to
  discuss their progress or any problems they might be having on a
  regular basis.
 \item
  Supervisors are to make themselves and their resources available to
  the people they are supervising whenever possible.
\end{enumerate}
