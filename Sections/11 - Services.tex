\section{Services}
\label{services}

\subsection{Publications}
\label{publications}

\subsubsection{Approval}
\label{approval}
All material published in the Plumbline, Frequency and Engineering
Handbook must first get unanimous approval from the Editorial Review
Committee (see MES Bylaws Section \ref{editorial-review-committee}).

\subsubsection{Purpose}
\label{purpose}
The purpose of an MES publication is:

\begin{enumerate}
 \item
  To provide a means of communication for the MES.
 \item
  To provide information to the engineering students at McMaster
  University.
 \item
  To provide news and feature coverage of campus and/or engineering
  related events.
 \item
  To provide a forum through which students can express their concerns
  or opinions.

\end{enumerate}

\subsubsection{Editorial Policy}
\label{editorial-policy}
Material considered for publication must be consistent with the above
purposes and the following editorial policy:

\begin{enumerate}
 \item
  Any material submitted to these publications is considered for
  publication if, in the opinion of the editors, it is:

  \begin{enumerate}
   \item
    A paid advertisement or classified listing.
   \item
    A feature dealing with engineering-related issues.
   \item
    A message from the MES Council or Executive.
   \item
    An editorial or letter to the editors.
   \item
    A feature coverage of a campus and/or engineering-related event.
   \item
    A message from Engineering Co-op and Career Services.
   \item
    A humorous piece for entertainment purposes.
  \end{enumerate}
 \item
  Material considered for publication will NOT be printed if, in the
  opinion of the Editor(s) or the Editorial Review Committee, it is:

  \begin{enumerate}
   \item
    RACIST. Racist material is any material deemed to explicitly or
    implicitly defame or discriminate against any person or group on the
    basis of their ethnic, national, or religious background.
   \item
    SEXIST. Sexist material is any material judged to explicitly or
    implicitly promote gender inequality, indicate sexual bias or imply
    discrimination on the basis of gender.
   \item
    HOMOPHOBIC. Homophobic material is any material judged to explicitly
    or implicitly defame or discriminate against any specific person or
    group on the basis of their (perceived) sexual orientation, or be
    based upon hatred and/or intolerance of homosexuals or homosexual
    culture.
   \item
    LIBELLOUS. Libellous material is any material that unfairly defames
    any person's character or reputation through innuendo or falsehood.
   \item
    PORNOGRAPHIC. Pornographic material is any material judged to be
    erotic or sexual nature intended to excite prurient feelings.
  \end{enumerate}
 \item
  The use of vulgar language will be limited to the Plumbline and the
  Engineering Handbook. Vulgar language will be limited to those
  colloquialisms commonly found in the average university student's
  vocabulary.
 \item
  All other areas not specifically covered by this policy are left to
  the discretion of the Editor(s) and the Editorial Review Committee.
 \item
  If the occasion arises where an article is printed and distributed
  which the Dean of Engineering deems unacceptable for reading, the
  paper must immediately be retracted. In this case, at least 80\% of
  the publication must be collected.
 \item
  Any complaints about the contents of one of the publications of the
  MES shall be handled in the following manner:

  \begin{enumerate}
   \item
    The MES VPC must be contacted and informed of the specific concerns
    and problems.
   \item
    All valid complaints will receive a response within two weeks and an
    invitation to meet with the MES President and the Editorial Review
    Committee.
   \item
    If the situation is not resolved at this point, a meeting will be
    set with the Dean of Engineering, the MES President, and the
    Editorial Review Committee to discuss appropriate actions.
   \item
    If at any point it is felt that a formal apology or any such action
    is warranted, one will be sent to the complainant and published in
    the next issue of the publication.

  \end{enumerate}
\end{enumerate}

\subsubsection{Plumbline}
\label{plumbline}
The Plumbline is the McMaster Engineering Society's humorous student
publication.

\begin{enumerate}
 \item
  Printing and Distribution

  \begin{enumerate}
   \item
    Printed in newspaper format, it may contain student pictures, jokes,
    etc. selected by the Plumbline Editor(s).
   \item
    Issue distribution should be scheduled at least twice a semester,
    pending time and budget constraints.
   \item
    The Plumbline Editor(s), Publications Editor, and the VPC should
    decide on the number of papers printed.
   \item
    The paper should be distributed through the DW Lounge.
   \item
    The entire contents of each Plumbline must be reviewed by the
    Editorial Review Committee to ensure the contents strictly fall
    within the guidelines of the Editorial Policy (see MES Bylaws
    Section \ref{editorial-review-committee} and
    \ref{editorial-policy}). % fix this lol - MES Policy Manual Section B.1.3, verify correct adjustment
   \item
    The Plumbline must be given to the Editorial Review Committee one
    week prior to being sent to print. The Plumbline can only be sent
    for print once the Plumbline Editor(s) have received a response from
    the majority of the Editorial Review Committee.
  \end{enumerate}
 \item
  Disclaimer Policy

  \begin{enumerate}
   \item
    The contents of the paper must include a disclaimer that explains to
    all readers that the articles are for entertainment purposes only,
    and are not intended to offend any person or party.
   \item
    A similar disclaimer should be given to sponsors to make them aware
    of the paper's contents.
   \item
    If student pictures or names are being used, the Plumbline Editor(s)
    must obtain the written permission of the student(s).

  \end{enumerate}
\end{enumerate}

\subsubsection{Frequency}
\label{frequency}
The Frequency is a serious publication that includes information
regarding upcoming academic and social events or announcements of the
MES and affiliated clubs and teams. Material from other groups can be
included, space permitting. Inappropriate pictures such as pictures of
substance consumption/abuse will not be printed in the Frequency.

\begin{enumerate}
 \item
  Publishment and Distribution

  \begin{enumerate}
   \item
    The Frequency is published quarterly (twice a semester) of the
    academic year.
   \item
    The number of copies printed is left at the discretion of the
    Frequency Editor(s), Publications Editor, and the VPC.
   \item
    Distribution should occur via the MES website, social media, and MES
    representatives.
   \item
    An additional supply of the Frequency should be placed in the DW
    Lounge, Engineering Alumni Office, Associate Dean's Office, and
    outside of Engineering Co-op and Career Services.

  \end{enumerate}
\end{enumerate}

\subsubsection{Engineering Handbook}
\label{engineering-handbook}
The Engineering Handbook is a publication primarily targeted to first
year engineering students with the intention of introducing them to
McMaster Engineering culture, while providing them with a humorous and
useful agenda booklet.

\begin{enumerate}
 \item
  Printing and Distribution

  \begin{enumerate}
   \item
    The Handbook Editor(s) shall ensure that the Handbook is ready for
    distribution during Welcome Week.
   \item
    The Handbook Editor(s), Publications Editor, and the VPC shall
    decide the number of Handbooks to be printed, provided that at least
    enough copies are printed to provide one to each first year student.
   \item
    Extra copies shall be distributed to upper level students.
  \end{enumerate}
 \item
  The Handbook should include the following components:

  \begin{enumerate}
   \item
    Front and back laminated covers, illustrated in colour
   \item
    MES Council email and position list
   \item
    MES Executive introductions
   \item
    Class schedule for each term
   \item
    Monthly calendars
   \item
    Daily planners with MES events printed on appropriate dates
   \item
    List and description of MES committees and sign up procedures
   \item
    Page for contact listings
   \item
    Jokes, poems, drawings, engineering traditions
   \item
    On and off-campus advertisements
   \item
    Coupons
   \item
    Annual MES award information
   \item
    Introduction to the MES
   \item
    List and description of all MES Awards
  \end{enumerate}
\end{enumerate}

\subsection{Website}
\label{website}
The MES website acts as a source of information for the McMaster
Engineering student body and those interested in our faculty. With that
in mind, the following points should be followed with respect to the
website.

\begin{enumerate}
 \item
  The Website Coordinator(s) is responsible for creating and maintaining
  the website.
 \item
  The Information Technology Coordinator(s) is responsible for resolving
  technical issues surrounding the website and ensuring website uptime.
 \item
  The language throughout the website should be clean and appropriate.
 \item
  There shall be no pictures of substance consumption/abuse or people
  committing inappropriate acts.
 \item
  Images that could be used for blackmail purposes will not be posted.
 \item
  If requested by any party, specified images must be removed promptly
  and without question.
 \item
  The website should have an up to date repository of all MES documents
  (Constitution, Bylaws, Policy Manual, meeting minutes, Plumbline,
  Frequency, Reports, etc.) and event news.
 \item
  Website editing access will be limited to the President, MES
  Executive, Website Coordinator(s), and Information Technology
  Coordinator(s). Further access may be granted at the discretion of the
  VPC.

\end{enumerate}

\subsection{MESsenger}
\label{messenger}
The MESsenger is an email list established to enhance communication to
MES members about MES events and activities, as well as other relevant
information that involves or is of particular interest to the
engineering student body.

\begin{enumerate}
 \item
  The VPC is responsible for compiling a monthly newsletter of MES
  activities, opportunities, club//team details and related information
  to be sent out to the list.
 \item
  The MESsenger will be sent out to all subscribed students on its email
  list, subscription is opt-in and may be cancelled at any time.
 \item
  The integrity of the MESsenger is of the utmost importance; the list
  must never be abused or used for any other reason than those stated
  above. The privacy of all subscribers to the list must be respected.
\end{enumerate}

\subsection{DW Lounge}
\label{dw-lounge}
The David Wilkinson Undergraduate Engineering (DW) Lounge is for the use
of undergraduate engineering students only. It is thus the
responsibility of the Society to maintain the room in an orderly way.

\subsubsection{Facilities and Services}
\label{facilities-and-services}

\begin{enumerate}
 \item
  The following shall be provided in the DW Lounge:

  \begin{enumerate}
   \item
    Microwave
   \item
    Couches, chairs, tables, whiteboards with markers
   \item
    Access to Project Magazine and Engineering Dimensions
   \item
    Mailboxes for MES Council members
  \end{enumerate}
 \item
  Rules of Conduct

  \begin{enumerate}
   \item
    Students are expected to behave appropriately.
   \item
    Vandalism is not permitted.
   \item
    Students are permitted to eat in the lounge, but are responsible for
    the clean up of their own food.
   \item
    Notices of these rules and regulations should be posted in the DW
    Lounge.
  \end{enumerate}
 \item
  All facilities shall be maintained and kept in working order by the DW
  Lounge Coordinator(s) (see MES Bylaws Section
  \ref{dw-lounge-coordinators}).
\end{enumerate}
\subsection{Drain}
\label{drain}

The Drain is the store of the MES. It is the main distribution outlet
for McMaster Engineering paraphernalia, tickets to events, and any other
items sold on behalf of the MES. The Drain Coordinator(s) are
responsible for daily operation of the Drain (see MES Bylaws Section
\ref{drain-coordinators}).

\subsubsection{Staffing Policies}
\label{staffing-policies}

\begin{enumerate}
 \item
  Hiring

  \begin{enumerate}
   \item
    It is the Drain Coordinator's responsibility to recruit responsible
    and reliable volunteers to staff the Drain.
   \item
    The Drain Coordinator(s) should select students from different years
    if possible.
   \item
    All students selected by the Drain Coordinator(s) must sign a
    contract prior to working in the Drain (see MES Policy Manual
    Section G.A). % appendix - wtf? G.A??? how tf is that a thing
   \item
    All Drain volunteers must hold a MES membership throughout the
    duration of their term, unless a financial need exemption is granted
    at the discretion of the Chief Returning Officer.
  \end{enumerate}
 \item
  Training

  \begin{enumerate}
   \item
    It is the sole responsibility of the Drain Coordinator(s) to educate
    and train the volunteers. There will be an employee training session
    offered by the Drain Coordinator(s) which outlines the procedures
    involved with selling items, and proper sales etiquette and time
    commitments.
  \end{enumerate}
 \item
  Staff Responsibilities

  \begin{enumerate}
   \item
    Each volunteer is responsible to appear for their shift at their
    scheduled time. If any volunteer cannot appear during their time,
    the volunteer must give at least 24 hours notice to the Drain
    Coordinator(s) outlining the reason they can not make it. If there
    is no reason given, the volunteer is given a warning. Volunteers may
    be removed from the staff if this is a recurring event at the
    discretion of the Drain Coordinator(s).
  \end{enumerate}
\end{enumerate}

\subsubsection{Theft Policy}
\label{theft-policy}
Regarding any theft of items from the Drain, the following actions must
be taken:

\begin{enumerate}
 \item
  All volunteers working that day should be informed of the theft.
 \item
  The Drain Coordinator(s) is responsible for finding the person or
  persons involved in the theft. If the guilty party is not found, the
  Drain Coordinator(s) must refer to the inventory check sheet.
 \item
  The Drain Coordinator(s) must report any thefts directly to the
  President and VPF
 \item
  To reduce the amount of theft that can occur, only \$200 is allowed to
  remain in the drain overnight. The Drain Coordinator(s) is responsible
  to coordinate with the VPF to put any additional money in the safe.
\end{enumerate}

\subsubsection{Pricing}
\label{pricing}
The goal of the Drain is not to make a tangible profit towards the MES,
but rather to provide a service to the engineering students. For this
reason, prices do not have to be increased from the wholesale value by a
large amount. It is left at the discretion of the Drain Coordinator(s)
and VPF how much to mark up or discount each item.

\subsubsection{Donations}
\label{drain-donations}

\begin{enumerate}
 \item
  When asked for a donation for a conference or event, the Drain
  Coordinator(s) are entitled to give a discount or gratuitous
  contribution towards the given cause up to a maximum value of \$20.
  The student or group requesting the donation must fill out a form (see
  MES Policy Manual Section G.H) outlining the reason why a donation
  should be given. % wtf ?
 \item
  It is at the discretion of the VPF whether or not to issue a discount
  or donation valued over \$20.

\end{enumerate}

\subsubsection{Advertising Policy}
\label{advertising-policy}
\begin{enumerate}
 \item
  All advertising must be developed and approved by the VPC.
 \item
  There must be no degradation of other faculties, professors, students
  or university staff.
 \item
  There is to be no false advertising. It should be indicated in the ad
  that a sale is valid ``while quantities last.''

\end{enumerate}

\subsubsection{Access Cards}
\label{access-cards}
Only the following people will have access cards to get into the Drain:

\begin{enumerate}
 \item
  President
 \item
  Vice President, Finance
 \item
  Vice President, Academic
 \item
  Vice President, Internal
 \item
  Drain Coordinator(s)
 \item
  The Co-Orientation Coordinators (From April of each year until 2 week
  after Welcome Week) % clerical
 \item
  Associate Vice President, Events
 \item
  Associate Vice President, Academic Resources

\end{enumerate}

\subsubsection{Student Organizations Merchandise}
\label{student-organizations-merchandise}
Engineering student organizations wishing to sell items through the
Drain must contact the Drain Coordinator(s) for authorization. This
service will be strictly limited to MES Clubs and Teams. Item sales are
separated into three categories: pre-orders \& event tickets, small
merchandise, and large merchandise.

\begin{enumerate}
 \item
  Merchandise pre-orders and event tickets are characterized by not
  requiring any physical merchandise to be kept in the Drain.

  \begin{enumerate}
   \item
    Student organizations wishing to sell merchandise pre-orders or
    event tickets through the Drain must send the Drain Coordinator(s) a
    detailed email describing the merchandise/event, pricing, number to
    be sold, and a link to a google form for customers to fill out (if
    necessary). This email must be sent a minimum of 2 weeks before
    sales begin.
   \item
    The Drain will then sell said merchandise pre-order or event ticket
    for a maximum of 2 weeks unless specific permission is granted by
    the Drain Coordinator(s).
   \item
    The Drain is not responsible for the distribution of the pre-ordered
    merchandise once it arrives.
   \item
    Revenue shall be reimbursed by the VPF no more than one month after
    sales finish.
  \end{enumerate}
 \item
  Small merchandise is characterized by requiring minimal storage space
  and includes patches, stickers, pins, lanyards, stickers, and other
  such items.

  \begin{enumerate}
   \item
    Student organizations wishing to sell small merchandise through the
    Drain must fill out the Drain Club Merchandise Form and send it to
    the Drain Coordinator(s) (see MES Policy Manual Section G.W).
   \item
    The merchandise will be placed in the display and sold the same way
    as all regular merchandise.
   \item
    Revenue shall be reimbursed by the VPF semi-annually until there are
    no more items remaining.
  \end{enumerate}
 \item
  Large merchandise is characterized by requiring a significant amount
  of storage/display space. This includes shirts, hoodies, hats, and
  other such merchandise.

  \begin{enumerate}
   \item
    Large merchandise will generally not be taken due to lack of storage
    and display space.
   \item
    Exceptions may be made for more ``general'' engineering merchandise
    - merchandise which is not specific to one single student
    organization. This is at the discretion of the Drain Coordinator(s)
    and VPF.
   \item
    If an exception is granted, the merchandise must be processed in the
    same fashion as small merchandise.
  \end{enumerate}
\end{enumerate}

\subsection{MES Office}
\label{mes-office}

\subsubsection{Office Use}
\label{office-use}

The MES Office is for the use of the following MES Council members only:

\begin{enumerate}
 \item
  All MES Executive members
 \item
  All Associate Vice Presidents
 \item
  Graphic Designers
 \item
  Social Media Coordinator(s)
 \item
  Website Coordinator(s)
 \item
  Frequency Editor(s)
 \item
  Plumbline Editor(s)
 \item
  Handbook Editor(s)
 \item
  Information Technology Coordinator(s)
 \item
  Drain Coordinator(s)
 \item
  The Co-Orientation Coordinators
 \item
  Equity and Inclusion Officer
 \item
  Chief Returning Officer
 \item
  Publications Editor
 \item
  Gerald Hatch Centre Student Coordinator(s)
 \item
  All Department Representatives
 \item
  All Program Representatives
 \item
  All First Year Representatives
\end{enumerate}

\subsubsection{Keys, Security and Auxiliary Access}
\label{keys-security-and-auxiliary-access}
\begin{enumerate}
 \item
  The MES office is secured with a keycard lock for which only the above
  MES Council members will be given keycard access to.
 \item
  It is the Responsibility of the President to assign and revoke keycard
  access to the MES Office.
 \item
  Under special circumstances, other MES Council members who request use
  of the office for MES Council activities may also be issued
  provisional access by one of the members listed in the MES Policy
  Manual Section B.6.1.
 \item
  Any MES members currently serving on the Executives of Engineering
  Student Societies' Council of Ontario or Canadian Federation of
  Engineering Students may be given limited access for Engineering
  Student Societies' Council of Ontario or Canadian Federation of
  Engineering Students business only. This request must be approved by
  the MES Executive.
 \item
  A contract must be signed by all office users to ensure that all rules
  pertaining to the MES Office are followed (see MES Policy Manual
  Section G.V) and must be submitted to the Administrator. % fix this lol

\end{enumerate}

\subsubsection{Etiquette and Housekeeping}
\label{etiquette-and-housekeeping}
\begin{enumerate}
 \item
  Each user of the office is responsible for cleaning up after
  themselves. This includes disposing of scrap paper and food, recycling
  appropriate materials, wiping off desks and the productions computer,
  and securely locking the door.
 \item
  At the beginning of the school year, at the end of each semester, and
  at the end of their term of office, the Administrator will conduct an
  inventory check of the office to ensure archived material is not lost
  and to check for theft. The MES is not accountable for items not
  belonging to MES.

\end{enumerate}

\subsubsection{Office Equipment}
\label{office-equipment}
\begin{enumerate}
 \item
  The MES Office contains one computer, printer, and stationary supply.
  The computer contains word-processing, spreadsheet, and design
  software for MES-related work including typing meeting minutes,
  creating posters, designing newspaper layouts, etc.
 \item
  The Information Technology Coordinator(s) is responsible for the
  installation and maintenance of office equipment.
 \item
  Only the Information Technology Coordinator(s) is authorized to
  install software on the computer.
 \item
  Any MES activities take priority over the use of the computer for
  personal business.
 \item
  Stationary is purchased as needed by the Administrator for the printer
  and for MES work only.

\end{enumerate}

\subsection{MES Trailer}
\label{mes-trailer}
The MES Trailer is property of the MES. It is available for use by any
recognized MES Group or MES member and shall be administered by the
Trailer Coordinator(s). No modifications (including applying decals to
the exterior) shall be made to the MES Trailer without prior consent
from the Trailer Coordinator(s) and the MES Executive.

\subsubsection{Bookings}
\label{bookings}

\begin{enumerate}
 \item
  All bookings must be made at least one week in advance.
 \item
  The Trailer Coordinator(s) will organize bookings.
 \item
  One person in the group booking the MES Trailer must sign a contract
  of liability.
 \item
  The group booking the MES Trailer must provide information about the
  driver, vehicle that will be towing the MES Trailer, and proof of
  insurance for towing a trailer.
 \item
  The group booking the MES Trailer is responsible for cleaning up any
  mess left in or on the MES Trailer. Failure to do so may result in a
  \$50 fine, at the discretion of the Trailer Coordinator(s) and the MES
  Executive.
 \item
  Any articles left in the MES Trailer become property of the MES if
  they are not claimed within one week of the MES Trailer being
  returned.

\end{enumerate}

\subsubsection{Keys}
\label{keys}
\begin{enumerate}
 \item
  There shall be three sets of keys to the MES Trailer, one of which
  will be held by the VPF and the other two will be held by the Trailer
  Coordinator(s).
 \item
  Keys can be picked up by the group who has the MES Trailer booked no
  more than two days before their scheduled booking period.
 \item
  All keys to the MES Trailer must be returned to the Trailer
  Coordinator(s) within two days after the scheduled booking period has
  ended.
 \item
  No additional copies of the MES Trailer keys may be made, except by
  the Trailer Coordinator(s) with written permission from the MES
  Executive. If illegal copies are found, the offending party will be
  subject to loss of privileges or other punishment as recommended by
  the MES Executive.

\end{enumerate}

\subsubsection{Use of Trailer}
\label{use-of-trailer}
All borrowers of the MES Trailer must comply with the MES Trailer Usage
Guidelines and sign the MES Trailer Use Contract and submit it to the
Trailer Coordinator(s) (see MES Policy Manual Section G.G). % fix this lol - more references

\subsubsection{Damage}
\label{damage}

\begin{enumerate}
 \item
  Any damage to the MES Trailer must be reported to the Trailer
  Coordinator(s) immediately.
 \item
  Groups or individuals borrowing the MES Trailer may be held
  responsible for any damage incurred during its use.
 \item
  Only the Trailer Coordinator(s) is authorized to perform or sanction
  any type of repair on the MES Trailer.
 \item
  If a group has not taken proper action in the event of damage and
  inconveniences another group's ability to follow through with their
  own booking, a minimum of \$100, or 10\% of the damage cost (whichever
  is greater) inconvenience fee will be imposed upon the group.

\end{enumerate}

\subsection{MES Equipment Rental}
\label{mes-equipment-rental}
The MES offers many different types of equipment for rental to MES
students and MES student groups. The full list of equipment is
summarised below: % fix this lol - just fucked up honestly

\begin{enumerate}
 \item
  Projector (see MES Policy Manual Section G.I)
 \item
  DSLR Camera (see MES Policy Manual Section G.J)
 \item
  Tent (see MES Policy Manual Section G.K)
 \item
  Wireless Radios (see MES Policy Manual Section G.L)
 \item
  Water Coolers \& Coffee Urns (see MES Policy Manual Section G.R)
 \item
  Button Maker (see MES Policy Manual Section G.T)
\end{enumerate}

\subsection{Storage Room}
\label{storage-room}

Only the President, VPI, VPF, Administrator, Drain Coordinator(s),
Culture Coordinator(s), and one of the Co-OCs shall have access to the
basement storage room (JHE-195A). The Co-OCs shall return this key by
the Friday following Welcome Week.

\subsection{McMaster Peer Tutoring Program}
\label{mcmaster-peer-tutoring-program}

The McMaster Peer Tutoring Program is a program developed to help
McMaster Engineering students excel in their studies at McMaster and to
create part-time employment opportunities for potential tutors. All
aspects of this program will be managed by the Student Success Centre
and the VPA.

\subsubsection{MES Tutoring Network}
\label{mes-tutoring-network}

\begin{enumerate}
 \item
  The McMaster Peer Tutoring Program is a program developed to provide
  McMaster students access to academic assistance through the provision
  of tutors for courses at McMaster at an affordable price. The MES is
  to maintain its partnership with the Student Success Centre unless the
  MES no longer benefits from the relationship. The MES Tutoring Network
  is a subset of the McMaster Peer Tutoring Program offered by the
  Student Success Centre. The MES Tutoring Network is designed to
  encourage peer-to-peer academic support by offsetting the cost of
  tutoring sessions between MES members. The VPA is responsible for
  maintaining the MES' relationship with the Student Success Centre to
  upkeep the Tutoring Network as long as the relationship is beneficial
  to the MES membership, and it follows the following expectations of
  the MES Tutoring Network:

  \begin{enumerate}
   \item
    The entirety of the monetary sponsorship of the program the MES
    provides shall be applied directly to reducing the cost of tutoring
    costs for MES members.
   \item
    Both the tutee and the tutor in a MES sponsored session must be MES
    members.
   \item
    There must exist a reasonable limit to the monetary value which any
    one tutee can take out of the system
  \end{enumerate}
 \item
  Online Coupons

  \begin{enumerate}
   \item
    The VPA will ensure that all MES members pay no more than \$5 per
    coupon, up to a limit of 10 per student per academic term
   \item
    More than one voucher may be redeemed per session with a tutor
   \item
    Tutors must be able to earn no less than \$15 per hour of tutoring
  \end{enumerate}
 \item
  Tutees

  \begin{enumerate}
   \item
    Tutees must be McMaster undergraduate students in the Faculty of
    Engineering
   \item
    Can only purchase vouchers for personal use
   \item
    May only purchase 10 vouchers per academic term
  \end{enumerate}
 \item
  Tutors

  \begin{enumerate}
   \item
    Must be registered with the MES to be recognized as an MES tutor
   \item
    Must have received a grade of 10 or higher in a course to be able to
    qualify to be a tutor for that course
   \item
    There is no limit to the number of redeemable vouchers for tutors
  \end{enumerate}

\end{enumerate}

\subsection{MES Textbook Library}
\label{mes-textbook-library}
The MES Textbook Library is a service provided by the MES to provide
McMaster undergraduate students in the Faculty of Engineering with
textbooks for free. The MES Textbook Library shall be run out of the
Drain and shall be maintained by the AVPAR.

\subsubsection{Operation}
\label{operation}

\begin{enumerate}
 \item
  The MES Textbook Library shall be run out of the Drain during regular
  operating hours.
 \item
  When an individual signs out a textbook, the Drain employee must
  record the individual's student number along with the identification
  of which textbook is signed out.
 \item
  All textbooks shall be returned to the Drain before the Drain's last
  operating hours of a semester.
 \item
  If a textbook is outstanding past the end of the semester, the
  individual who signed out the account shall be charged a \$100 fine to
  their student account, unless otherwise arranged with the AVPAR.

\end{enumerate}

\subsection{MES Academic Workshops}
\label{mes-academic-workshops}

\subsubsection{MES-Organized
 Workshops}
\label{mes-organized-workshops}
\begin{enumerate}
 \item
  The MES runs academic and technical workshops for general first year
  and second year engineering courses. The AVPAR is responsible for
  organizing workshops, finding workshop leads (upper year students),
  and paying them.
 \item
  Workshop leads shall be paid at a rate of \$27/hour, up to a maximum
  of 3 hours per workshop.
 \item
  Workshop leads must complete Appendix Y and submit it to the AVPAR in
  order to be reimbursed.

\end{enumerate}

\subsubsection{Department/Program-Society-Organized Academic
 Workshops}
\label{departmentprogram-society-organized-academic-workshops}
\begin{enumerate}
 \item
  Departments may run their own academic workshops for
  department-specific courses. Department societies are responsible for
  organizing workshops and finding workshop leads, and paying them.
 \item
  Workshop leads shall be paid at a rate of \$27/hour, up to a maximum
  of 3 hours per workshop and 1 help session lead per workshop.
  Departments may send a request for more funding to the AVPAR and/or
  VPF if they have more than 1 lead at a help session, however extra
  funding is not guaranteed.
 \item
  MES department/program representatives are responsible for ensuring
  that their workshop leads complete Appendix Y and for submitting it to
  the AVPAR in order for their department/program society to be
  reimbursed.
\end{enumerate}

\subsection{Coveralls}
\label{coveralls}
Coveralls are worn by undergraduate engineering students to represent
the Faculty of Engineering. The coveralls are a well-known symbol within
the University, the Hamilton area, and engineering schools across the
province and country. The coveralls have a complex, deep-rooted history.
The coverall approval process ensures that the coveralls are worn by
those who understand the significance of the coverall and the
responsibility that comes with wearing the coverall.

\subsubsection{Coverall Oversight Committee}
\label{coverall-oversight-committee}

\begin{enumerate}
 \item
  The Coverall Oversight Committee is responsible for:

  \begin{enumerate}
   \item
    Ensuring the coveralls are positively representing the Faculty of
    Engineering
   \item
    Reviewing and approving Coverall Sponsor applications
   \item
    Reviewing and approving Coverall Event applications
   \item
    Reviewing and approving Coverall applications
   \item
    Rescinding Coverall Sponsor and Coverall Event approval
   \item
    Evaluating infringements and consequences for individuals who break
    the Sponsor Contract, Coverall Usage Contract, and Coverall Code of
    Conduct.
  \end{enumerate}
 \item
  The Coverall Oversight Committee will be chaired by the AVP Clubs.
 \item
  The voting membership of the Coverall Oversight Committee shall
  consist of:

  \begin{enumerate}
   \item
    President
   \item
    Vice-President, Student Life
   \item
    Vice-President, External
   \item
    Co-Orientation Coordinators
    \begin{enumerate}
     \item
      The term is their Welcome Week calendar year (January - December)
    \end{enumerate}
   \item
    Equity, Diversity, and Inclusion Officer
  \end{enumerate}
 \item
  All decisions of the committee must be unanimous.
\end{enumerate}

\subsubsection{Responsibilities}
\label{coverall-responsibilities}
\begin{enumerate}
 \item
  The Vice-President, External is responsible for:
  \begin{enumerate}
   \item
    Overseeing and being ultimately responsible for the coveralls.
   \item
    Track the inventory of coveralls and organize future orders of
    coveralls.
   \item
    Collaborate with the Drain Coordinators on the purchase and
    distribution of coveralls.
   \item
    Coordinating sizing, pick-up, and payment with approved applicants
    during regular Drain operating hours in order for the Drain
    volunteers to facilitate payment.
   \item
    Maintaining and updating the Coverall Sponsor, Coverall Event, and
    Coverall Application forms.
   \item
    Collecting a list of individuals attending the Coverall Event from
    the Coverall Sponsor.
   \item
    Ensuring the voting members of the Coverall Oversight Committee
    review applications on a regular basis.
   \item
    Communicating application status with Coverall Sponsors and the
    application status of their Coverall Event(s).
   \item
    Maintaining a list of club executives, vetting the eligibility of
    Coverall Sponsor applicants, and acting as the Coverall Sponsor for
    Student Organization leadership.
   \item
    Communicating application status with Coverall Applicants.
   \item
    Oversee the Chair and delegate responsibilities to the Chair at
    their discretion.
  \end{enumerate}
 \item
  The Chair is responsible for:
  \begin{enumerate}
   \item
    Organizing meetings to further discuss applications, where
    necessary.
   \item Executing responsibilities as delegated by the Vice-President,
    External.
  \end{enumerate}
\end{enumerate}
\subsubsection{Coverall Sponsors}
\label{coverall-sponsors}

\begin{enumerate}
 \item
  Coverall Sponsors are individuals in the engineering community who act
  as sponsors for individuals who are applying for coveralls.
 \item
  Student leaders in MES Groups, Teams, Program Societies, Committees,
  and Affiliates (hereafter referred to as ``Student Organizations'')
  may apply to become a Coverall Sponsor.

  \begin{enumerate}
   \item
    Only the President, Lead, Principal Investigator, or individual of
    equivalent title may apply.
   \item
    Sub-team leads or members of the Student Organization's executive
    cannot apply
   \item
    If multiple individuals hold the aforementioned position, only one
    individual can apply.
  \end{enumerate}
 \item
  The Coverall Sponsor application will consist of questions determined
  at the discretion of the Coverall Oversight Committee.
 \item
  All Coverall Sponsor applicants must submit a signed Coverall
  Sponsorship Usage Contract (Appendix Z) with their application.
 \item
  Coverall Sponsor applicants will be approved or denied at the
  discretion of the Coverall Oversight Committee.
 \item
  Once approved as a Coverall Sponsor, the individual is eligible as a
  Coverall Sponsor until August 31st. Sponsors need to be reinstated
  each year they are eligible to be sponsors.
  \begin{enumerate}
   \item
    Student Organizations with a transfer of leadership throughout the
    year may change Coverall Sponsors through the new leadership
    applying to be a Coverall Sponsor
  \end{enumerate}
 \item
  Being approved as a Coverall Sponsor is independent of the Coverall
  Application; Coverall Sponsors must complete the Coverall Application
  if they are interested in wearing their own coveralls.
 \item
  The Co-Orientation Coordinators are exempt from the Coverall Sponsor
  process, by virtue of their position and the nature of their hiring.
 \item
  The President is exempt from the Coverall Sponsor process, by virtue
  of their position.

\end{enumerate}
\subsubsection{Coverall Events}
\label{coverall-events}
\begin{enumerate}
 \item
  Coverall Events are events where the coverall will represent the
  Faculty of Engineering in a positive light.
 \item
  Coverall Event applications must be submitted by an approved Coverall
  Sponsor.

  \begin{enumerate}
   \item
    Coverall Events for a Student Organizations must be submitted by the
    President, Lead, Principal Investigator, or individual of equivalent
    title.
  \end{enumerate}
 \item
  Coverall Event applications are required for all events where
  coveralls will be worn.
 \item
  Once approved as a Coverall Event, the event is approved for future
  years; the event does not need to be resubmitted each year.

  \begin{enumerate}
   \item
    The Coverall Sponsor for the event must be the current President,
    Lead, Principal Investigator, or individual of equivalent title.
  \end{enumerate}
 \item
  Examples of eligible Coverall Events include, but are not limited to:

  \begin{enumerate}
   \item
    Welcome Week
   \item
    ESSCO and CFES Conferences
   \item
    Team competitions
   \item
    Contingents at a parade
  \end{enumerate}
 \item
  The Coverall Event application will consist of questions determined at
  the discretion of the Coverall Oversight Committee.
 \item
  Coverall Events will be approved or denied at the discretion of the
  Coverall Oversight Committee.

\end{enumerate}

\subsubsection{Coverall Applications}
\label{coverall-applications}
\begin{enumerate}
 \item
  The Coverall Application is required for individuals who are
  interested in obtaining their own coveralls to represent the Faculty
  of Engineering at a Coverall Event (hereafter referred to as
  ``Coverall Applicants'').
 \item
  The Coverall Application Form will consist of the following:

  \begin{enumerate}
   \item
    Personal Information

    \begin{enumerate}
     \item
      Full Name
     \item
      McMaster Email
     \item
      Starting year at McMaster
     \item
      Program
     \item
      Proof of Enrolment
    \end{enumerate}
   \item
    Coverall Information

    \begin{enumerate}
     \item
      Coverall Event
     \item
      Sponsor Name
     \item
      Signed Coverall Usage Contract (Appendix AA)
     \item
      Signed Coverall Code of Conduct (Appendix AB)
    \end{enumerate}
   \item
    Proof of Training

    \begin{enumerate}
     \item
      Accessibility for Ontarians with Disabilities Act (AODA) and
      University Health \& Safety (UHS) Trainings

      \begin{enumerate}
       \item
        AODA and Human Rights Code
       \item
        Health and Safety Orientation
       \item
        Violence and Harassment Prevention
      \end{enumerate}
     \item
      It Takes All of Us (CONSENT 1A00)
     \item
      Responding to Disclosures on Campus
     \item
      More Feet on the Ground
    \end{enumerate}
  \end{enumerate}
 \item
  The Coverall Application may consist of additional questions and
  training determined at the discretion of the Coverall Oversight
  Committee.
 \item
  Coverall Applications will be cross-referenced with a list of
  individuals attending the Coverall Event
 \item
  Coverall Applications will be approved or denied at the discretion of
  the Coverall Oversight Committee.
 \item
  Once approved, the Coverall Applicant will be contacted to coordinate
  sizing, pick-up, and payment.
 \item
  Engineering Welcome Week Representatives are exempt from the Coverall
  Application process, by virtue of the Welcome Week Representative
  recruitment process and the Student Success Centre Welcome Week
  Representative training.
  \begin{enumerate}
   \item
    Engineering Welcome Week Representatives are bound to all other
    regulations other than the Coverall Application process.
  \end{enumerate}
\end{enumerate}
