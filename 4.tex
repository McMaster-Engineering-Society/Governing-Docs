% % !TEX program = xelatex

% \documentclass{article}
% \usepackage[letterpaper, portrait, margin=1in]{geometry}
% \usepackage{hyperref}
% \usepackage{alphalph}

% \setcounter{secnumdepth}{5}
% \renewcommand{\labelenumi}{\alphalph{\value{enumi}})}
% \renewcommand{\labelenumii}{\alph{enumii}.}
% \renewcommand{\labelenumiii}{\roman{enumiii}.}

% \begin{document}

\hypertarget{clubs-groups-teams-and-affiliates}{%
 \section{CLUBS: GROUPS, TEAMS, AND AFFILIATES}
 \label{clubs-groups-teams-and-affiliates}}

\hypertarget{program-societies}{%
 \subsection{Program Societies}
 \label{program-societies}}
\begin{enumerate}
 \item
  A program society shall be a fully recognized organization of McMaster
  undergraduate students, in the Faculty of Engineering. Societies shall
  be affiliated with either a specific engineering program.
 \item
  The MES recognizes the following program societies:

  \begin{enumerate}
   \item
    Bachelor of Technology Association (BTA)
   \item
    Chem Eng Club (CEC)
   \item
    Civil Engineering Society (CSS)
   \item
    Computer Science Society (CSS)
   \item
    Electrical and Computer Engineering Society (ECES)
   \item
    Engineering and Society Students' Association (ESSA)
   \item
    Engineering Physics Society (EPS)
   \item
    iBioMed Society (iBMS)
   \item
    Materials Science and Engineering Society (MSES)
   \item
    McMaster Engineering and Management Society (MEMS)
   \item
    McMaster Mechatronics Society (MMS)
   \item
    McMaster Society of Mechanical Engineering (MSME)
   \item
    Software Engineering Society (SES)
  \end{enumerate}
 \item
  All societies shall report their financial statement and receive
  funding as outlined in the Clubs Financial Policy (see MES Bylaws
  Section \ref{clubs-and-teams-funding-policy})
 \item
  All society presidents will be members of the Club Leaders Committee
  (see MES Bylaws Section \ref{club-leaders-committee}).
 \item
  New MES Program Societies must put forth a motion for recognition by
  the MES at a General Meeting.

\end{enumerate}

\hypertarget{groups}{%
 \subsection{Groups}
 \label{groups}}
\begin{enumerate}
 \item
  A Group shall be an organization composed of at least 75\% McMaster
  undergraduate students in the Faculty of Engineering.
 \item
  A Group may receive funding from the MES by following the procedures
  set out in MES Bylaws Section \ref{clubs-and-teams-funding-policy}.
 \item
  The Groups that are officially recognized by the MES are:

  \begin{enumerate}
   \item
    Brewing Club, McMaster
   \item
    Deep Space Analogue Research Expedition, McMaster (DARE)
   \item
    DeltaHacks
   \item
    Design League, McMaster (MDL)
   \item
    Engineering Musical, McMaster (MEM)
   \item
    Engiqueers, McMaster (EQ)
   \item
    Engineers Without Borders (EWB)
   \item
    Google Developer Student Clubs, McMaster (GDSC)
   \item
    HealthHatch
   \item
    Heavy Construction Student Chapter, McMaster (HCSC)
   \item
    Institute of Electrical and Electronics Engineers, McMaster Student
    Branch (IEEE)
   \item
    Institute of Transportation Engineers Student Chapter, McMaster
    (ITE)
   \item
    Leading Edge, McMaster
   \item
    Mac Eng Rugby (Fireball Rugby)
   \item
    Mars Rover Team, McMaster (MMRT)
   \item
    McMaster Engineering Hockey Club
   \item
    McMaster Engineering Custom Vehicle Team (MECVT)
   \item
    McMaster Society for Engineering Research (MacSER)
   \item
    Medical Engineering Design Team, McMaster (MED-T)
   \item
    National Society of Black Engineers, McMaster Chapter (NSBE)
   \item
    North American Young Generation in Nuclear, McMaster (NAYGN)
   \item
    Pumpkin Chunkin Club, McMaster Engineering (MPC)
   \item
    Start Coding, McMaster
   \item
    Women in Engineering, McMaster (WiE)
  \end{enumerate}
\end{enumerate}

\hypertarget{teams}{%
 \subsection{Teams}
 \label{teams}}
\begin{enumerate}
 \item
  Teams shall be a subset of MES Groups. A Group is considered to be a
  team if they are a non-sporting organization composed of at least 75\%
  McMaster undergraduate students in the Faculty of Engineering which
  competes at external events as representatives of McMaster.
 \item
  In order to receive team status, teams must have competed in at least
  one external event the previous year, or be approved by the MES
  council for extenuating circumstances. Otherwise, the club's will be
  transferred to group status.
 \item
  Teams that are officially recognized by the MES are:

  \begin{enumerate}
   \item
    Baja Racing, McMaster
   \item
    Chem E Car, McMaster
   \item
    Competitive Programming, McMaster (MCP)
   \item
    Concrete Toboggan, McMaster (MECTT)
   \item
    EcoCar, McMaster
   \item
    Formula Electric, McMaster (MACFE)
   \item
    Mechanical Contractors Association Hamilton Niagara, McMaster
    Student Chapter (MCAHN)
   \item
    RoboMaster Team, McMaster
   \item
    Rocketry, McMaster
   \item
    Seismic Design Team, McMaster (MSDT)
   \item
    Solar Car, McMaster
   \item
    Steel Bridge Team, McMaster (MSBT)
  \end{enumerate}
 \item
  Team projects should involve the application of engineering design
  concepts.

\end{enumerate}

\hypertarget{affiliates}{%
 \subsection{Affiliates}
 \label{affiliates}}
\begin{enumerate}
 \item
  An affiliate shall be an organization composed of at least 75\%
  McMaster undergraduate students in the Faculty of Engineering.
 \item
  An affiliate is a collection of people who wish to be recognized by
  the MES and do not require funding.
 \item
  Affiliates are a stepping stone for new clubs to eventually reach
  ratification as a group or team.
 \item
  Affiliates that are officially recognized by the MES are:

  \begin{enumerate}
   \item
    Aerospace Team, McMaster (MAST)
   \item
    American Indian Science and Engineering Society, McMaster Chapter
    (AISES)
   \item
    Autoplow, McMaster
   \item
    Blockchain Club, McMaster
   \item
    Concrete Canoe, McMaster
   \item
    Engineering Jazz Band, McMaster
   \item
    Engineers with Disabilities, McMaster
   \item
    Mac Quantum Club
   \item
    MacSmiths
   \item
    McMaster Interdisciplinary Satellite Team (MIST)
   \item
    McMaster Energy Association (MEA)
   \item
    Sumobot, McMaster
  \end{enumerate}
\end{enumerate}

\hypertarget{recognized-clubs}{%
 \subsection{Recognized Clubs}
 \label{recognized-clubs}}
\begin{enumerate}
 \item
  A recognized club is a club associated with the MES that:

  \begin{enumerate}
   \item
    Must have a STEM-related focus that directly relates to an
    engineering discipline.
   \item
    Cannot be used as a stepping stone to MES Club ratification
   \item
    Cannot receive MES funding
   \item
    Is ratified by another organization within McMaster
   \item
    Does not need to have 75\% McMaster undergraduate students in the
    Faculty of Engineering
  \end{enumerate}
 \item
  Recognized clubs may:

  \begin{enumerate}
   \item
    use MES branding in promotional materials and advertisements
   \item
    participate in ClubsFest and other MES Club Events.
   \item
    collaborate with AVPC for help planning events
  \end{enumerate}
 \item
  Recognized clubs are not required to give accountability presentations
  at SAGM.
 \item
  To become a recognized club, the organization must present a motion at
  an MES Council meeting.
\end{enumerate}

\hypertarget{ratification-process}{%
 \subsection{Ratification Process}
 \label{ratification-process}}
\begin{enumerate}
 \item
  To become an MES Affiliate, you must email the AVPC your intent to
  become an MES Affiliate.
 \item
  After meeting with the AVPC, all intending MES Affiliates must put
  forth a motion for affiliation at a Council Meeting. Council will vote
  on the motion and the AVPC will assist you with the next steps
  whatever the result!
 \item
  To become an MES Group, you must be an MES Affiliate for at least one
  year, then let the AVPC know your intent to apply to be an MES Group.
 \item
  Each MES Affiliate intending to become an MES Group must present a
  motion to MES Council to be recommended to the General Meeting. Upon a
  majority vote from Council, the MES Affiliate must submit a motion for
  ratification at the second General Meeting in a given academic year
  (See MES Bylaws Section \ref{general-meetings-sagm}).

  \begin{enumerate}
   \item
    On a majority vote at the General Meeting, the MES Affiliate will be
    ratified as a Group recognized by the MES.
   \item
    Motions may be put forth at the first General Meeting in a given
    academic year only by direct permission from the AVPC.
  \end{enumerate}
 \item
  Long-established clubs may be able to bypass the affiliation stage and
  become directly ratified as an MES Group.

  \begin{enumerate}
   \item
    This follows the same ratification process for MES Affiliates,
    wherein Council will motion whether to send the club to the Semi
    Annual General Meeting for ratification.
  \end{enumerate}

\end{enumerate}

\hypertarget{de-ratification-process}{%
 \subsection{De-Ratification Process}
 \label{de-ratification-process}}
If a club is not performing or achieving certain standards, they may be
de-ratified.

\begin{enumerate}
 \item
  The de-ratification process will occur at the Winter General Meeting.
 \item
  The AVPC shall be responsible for compiling a list of clubs that shall
  undergo the de-ratification process. A motion for de-ratification may
  be proposed at the discretion of the AVPC, but reasons could include
  that the club:

  \begin{enumerate}
   \item
    Does not present at either General Meeting in that year, nor
    explains their absence to council
   \item
    Does not attend ClubsFest, or any outreach events during the
    academic year
   \item
    Does not host events or maintain any social media presence during
    the academic year
  \end{enumerate}
 \item
  The AVPC will be responsible for communicating with clubs that may be
  facing de-ratification.
 \item
  The AVPC will present to Council all clubs that shall under-go the
  de-ratification process, and, upon approval from Council, these clubs
  shall be presented and voted for de-ratification at the Winter General
  Meeting.
 \item
  Groups, Teams, and Affiliates may all be subject to the
  de-ratification process
 \item
  Recognized clubs will be deratified if they are disbanded by the
  organization they are ratified by.
\end{enumerate}
% \end{document}